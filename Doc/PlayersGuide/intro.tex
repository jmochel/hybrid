\chapter{Quick and Dirty Intro to the mechanics}

\section{Rolling dice}

There is one main type of roll in \SH\ . The roll is 
made with percentile dice against a Success Chance (SC). If the roll 
is under or equal to the Success Chance, then the roll is successful.
If it is greater than the Success Chance then it is a failure.

The Success Chance is the percentage chance to perform a given task This SC is
determined from the statistics and skill of the character 
and modified based on the difficulty of the task. The most common 
modifier is called a Difficulty Factor (DF). This is a number that typically 
ranges from -10 to +10.

$ {Success Chance} = 3 \times {The\ Stat} + Rank \times 4 + {Difficulty\ of\ Task} \times 5 $

As an example if the player character has a Physical Strength (PST) of 15, a rank in 
weight lifting of 4 and is trying to lift half his weight in mass ( a Difficulty Factor or DF of -5). 
This means the Success Chance (SC) is $ SC = 3 \times PST + Rank \times 4 + DF \times 5 $
In the case of the weightlifting this means $ SC = 3 \times 15 + 4 \times 4 + -5 \times 5  = 36\% $

\subsection{Open-ended Rolls}

The range of die rolls is 1-100. If you roll 00 (a 100) you roll again add
\index{Open-ended Rolls}
the second roll to the first to get our total.

\subsection{Evaluating Success and Failure}

When percentile dice are rolled and the result is under the success chance, that
is a normal success. When the rolled number is significantly lower than the needed 
roll there is a chance the action may have a greater than normal success. This is 
called a ``Critical'' success. Table \ref{Table:CriticalSuccess} on page on
\index{Critical Success}
page~\pageref{Table:CriticalSuccess} describes the rolls needed.

As an example. If a character needed to roll a 40 or under to hit a target 
with a rock and they rolled under $ 1/2 $ of 40 then they will do 1.25
times the damage they would normally do.

In the case of very poor rolls there is a chance that the roll  
may be a critical failure This is caused by rolling 50 above your success chance or 
rolling above by $ 1/2 $ the success chance of the action, whichever is greater.

So someone with a success chance of 90 needs to roll a 140 or higher to
critically miss while someone with a success chance of 120 needs to roll a
180 or higher to critically miss.

To determine the severity of the critical failure roll against the 
amount missed by as a success chance and compare the result to table
\index{Critical Failure}
\ref{Table:CriticalFailure} on page~\pageref{Table:CriticalFailure}.

\begin{table}[hb]
\centering
\caption{Success Table}
	\begin{tabular}{||r|r||r|l|} \hline
	SN		&	SN	& EN 	& Subjective Result \\ \hline
	-200	& -176	& -3.0 	& \\
    -175	& -151	& -2.5 	& \\
	-150	& -126  & -2.0 	& \\
	-125   	& -101	& -1.5 	& Amazing Failure \\
	-100   	& -076	& -1.0 	& Notable Failure \\
	-075   	& -051	& -0.5 	& Solid Failure \\
	-050   	& -001	& +0.0 	& Normal Failure \\
	-000   	& -000	& +1.0 	& Near  Failure \\
	+001   	& +050	& +1.0 	& Normal Success \\
	+051   	& +075	& +1.5 	& Solid Success \\
	+076   	& +100	& +2.0 	& Notable Success \\
	+101   	& +125	& +2.5 	& Amazing Success \\
	+126   	& +150	& +3.0 	& \\  
	+151	& +175  & +3.5 	& \\
    +176	& +200	& +4.0 	& \\  \hline
	\end{tabular}
\end{table}

% FILE Ease Factor 
% REF 

\begin{figure}[h]
\begin{tabular}{llllllllllllllllll} \hline
	& \multicolumn{17}{c}{Ease Factor} \\ \hline \hline
SB  &-2   &-1    &0    &1    &2    &3    &4    &5    &6    &7    &8    &9   &10   &11   &12   &13 \\ \cline{2-18}
 1  &-28  &-14    &0    &1    &2    &3    &4    &5    &6    &7    &8    &9   &10   &11   &12   &13 \\
 2  &-26  &-13    &0    &2    &4    &6    &8   &10   &12   &14   &16   &18   &20   &22   &24   &26 \\
 3  &-24  &-12    &0    &3    &6    &9   &12   &15   &18   &21   &24   &27   &30   &33   &36   &39 \\
 4  &-22  &-11    &0    &4    &8   &12   &16   &20   &24   &28   &32   &36   &40   &44   &48   &52 \\
 5  &-20  &-10    &0    &5   &10   &15   &20   &25   &30   &35   &40   &45   &50   &55   &60   &65 \\
 6  &-18   &-9    &0    &6   &12   &18   &24   &30   &36   &42   &48   &54   &60   &66   &72   &78 \\
 7  &-16   &-8    &0    &7   &14   &21   &28   &35   &42   &49   &56   &63   &70   &77   &84   &91 \\
 8  &-14   &-7    &0    &8   &16   &24   &32   &40   &48   &56   &64   &72   &80   &88   &96  &104 \\
 9  &-12   &-6    &0    &9   &18   &27   &36   &45   &54   &63   &72   &81   &90   &99  &108  &117 \\
10  &-10   &-5    &0   &10   &20   &30   &40   &50   &60   &70   &80   &90  &100  &110  &120  &130 \\
11  &-8   &-4    &0   &11   &22   &33   &44   &55   &66   &77   &88   &99  &110  &121  &132  &143 \\
12  &-6   &-3    &0   &12   &24   &36   &48   &60   &72   &84   &96  &108  &120  &132  &144  &156 \\
13  &-4   &-2    &0   &13   &26   &39   &52   &65   &78   &91  &104  &117  &130  &143  &156  &169 \\
14  &-2   &-1    &0   &14   &28   &42   &56   &70   &84   &98  &112  &126  &140  &154  &168  &182 \\
15  &0    &0    &0   &15   &30   &45   &60   &75   &90  &105  &120  &135  &150  &165  &180  &195 \\ \hline
\end{tabular}
\caption{Ease Factor versus Stat Basis}
\end{figure}     

     
     
     
     
     
     
     
     
     
     
     
     
     
     
     

