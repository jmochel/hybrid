\chapter{Weapon Formats}

\section{Melee Weapons}

\begin{figure}[hb]
\centering
\caption{Melee Weapon Format}
	\begin{description}
		\item[Name]
		\item[Type (T)]
		\item[Length (L)]
		\item[Mass (M)]
		\item[Speed (S)]
		\item[Accuracy (A)]
		\item[Damage Multiplier (DM)]
		\item[Technological Index (TI)]
		\item[Tech Level Breakpoints]
		(MASS-LENGTH-ACC-DM)
		\item[Description]
		\item[Comments]
	\end{description}
\end{figure}

\section{Format Explanation}

\begin{description}
	\item[Name]
	Self Explanatory
	\item[Type (T)]
	Whether the weapon does (most commonly) Crushing, Slicing, or Piercing
	\item[Length (L)]
	The overall size in centimeters of a weapon.
	\item[Mass (M)]
	The mass of the unit in kilograms
	\item[Speed (S)]
	The time it takes, in pulses, to use the weapon in a combat action.
	\item[Accuracy (A)]
	Any additional bonus the weapon gets to hit for its workmanship or
	materials. This is usually very small.
	\item[Damage Multiplier (DM)]
	The multiple times the users base that is used to produce a characters
	damage.
	\item[Technological Index (TI)]
	The tech level in which the weapon made its 1st appearance.
	\item[Tech Level Breakpoints:]
	(MASS-LENGTH-ACC-DM)
\end{description}

\clearpage
\section{Hand to Hand Techniques}

\begin{figure}[hb]
\centering
\caption{Hand to Hand Technique Format}
	\begin{description}
		\item[Name]
		\item[Type]
		\item[Speed]
		\item[Damage Multiplier]
	\end{description}
\end{figure}

\section{Format Explanation}

\begin{description}
	\item[Name]
	Self Explanatory
	\item[Type (T)]
	Whether the weapon does (most commonly) Crushing, Cutting, or Piercing
	\item[Speed (S)]
	The time it takes, in pulses, to use the technique in a combat action.
	\item[Damage Multiplier (DM)]
	The multiple times the users base that is used to produce a characters
	damage.
\end{description}

\clearpage
\section{Missile and Projectile Weapon}

\begin{figure}[hb]
\centering
\caption{Missile and Projectile Weapon Description Format}
	\begin{description}
		\item[Name]
		\item[Configuration (Cfg)]
		\item[Action (Act)]
		\item[Type (T)]
		\item[Caliber (Cal)]
		\item[Energy (NRG)]
		\item[Length (L)]
		\item[Mass (M)]
		\item[Speed (S)]
		\item[Accuracy (A)]
		\item[Accuracy Decrement (AD)]
		\item[Damage (D)]
		\item[Damage Decrement  (DD)]
		\item[Technological Index (TI)]
		\item[Tech Level Breakpoints]
		(MASS-LENGTH-WA/WAD-WD/WDD)
	\end{description}
\end{figure}

\section{Format Explanation}

\begin{description}
	\item[Name]
	The most common and correct names are given. In the cases of weapons
	that are very common in ther effects, only the most notable will be
	described.
	\item[Configuration (Cfg)]
	The physical form the weapon has. Typically : Heavy Rifle, Rifle, 
	Carbine, Pistol 
	\item[Action (Act)]
	The type of trigger pull vs firing mechanism. Typically Single Shot, Auto Burst,
	Auto, Semi Auto, Revolver.
	\item[Type (T)]
	Chemical Propellant, CW Laser, Pulse Laser, Focused Sonic, Rocket, Gauss
	\item[Caliber (Cal)]
	The Apeture of the weapon
	\item[Energy (NRG)]
	The amount of energy the weapon delivers.
	\item[Length (L)]
	The overall size in centimeters of a weapon.
	\item[Mass (M)]
	The mass of the unit in kilograms
	\item[Speed (S)]
	The time in which the weapon can be brought to bear on a target. This is
	the normal time involved in bring a weapon to bear on a target and needs
	to be doubled if the weapon is being brought to bear on any target for
	the first time ( ie brought to the aiming position ) and needs to be tripled
	when the weapon is being drawn from a holster.
	\item[Accuracy (A)]
	Any ad\-di\-tional bonus to the chance to hit the weapon may have as a
	result of its design.
	\item[Accuracy Decrement (AD)]
	A measurement of how much the to hit chance of the weapon is lowered for
	a given amount of distance.
	\item[Damage (D)]
	The base amount of dam\-age done by a given weapon
	\item[Damage Decrement  (DD)]
	How much the damage of a given weapon is lowered for a given amount of
	distance.
	\item[Technological Index (TI)]
	The Technological Index in which the weapon made first 1st appearance.
	\item[Tech Level Breakpoints]
	(MASS-LENGTH-WA/WAD-WD/WDD)
\end{description}

\clearpage
\section{Thrown Weapon Format}

\begin{figure}[hb]
\centering
\caption{Thrown Weapon Description Format}
	\begin{description}
		\item[Name]
		\item[Length (L)]
		\item[Mass (M)]
		\item[Accuracy (A)]
		\item[Accuracy Decrement (AD)]
		\item[Damage Multiplier (DM)]
		\item[Damage Decrement (DD)]
		\item[Technological Index (TI)]
		\item[Technological Level Breakpoints:]
		(MASS | LENGTH | WA/WAD | WDM/WDD)
	\end{description}
\end{figure}

\section{Format Explanation}

\begin{description}
	\item[Name]
	Self Explanatory
	\item[Length (L)]
	The overall size in centimeters of a weapon.
	\item[Mass (M)]
	The mass of the unit in kilograms
	\item[Accuracy (A)]
	The additional bonus the weapon gets to hit for its workmanship or
	materials. This is usually very small.
	\item[Accuracy Decrement (AD)]
	The decrement per a given portion of the range that the weapon travels through
	\item[Damage Multiplier (DM)]
	The multiple times the entitie's PSE base that is used to produce a characters
	damage.
	\item[Damage Decrement (DD)]
	\item[Technological Index (TI)]
	The Technological Index in which the weapon made its 1st appearance.
	\item[Tech Level Breakpoints:]
	(MASS-LENGTH-WA/WAD-WDM/WDD)
\end{description}

\clearpage
\section{Explosive Description Format}

These are specific to blocks of explosive and weapons whose main 
function is as an explosive device. Shrapnel is treated as a
projectile weapon.

\begin{figure}[hb]
\centering
\caption{Explosive Description Format}
	\begin{description}
		\item[Name]
		\item[Type ]
		\item[Damage (DM)]
		\item[Damage Decrement (DD)]
	\end{description}
\end{figure}

\section{Format Explanation}

\begin{description}
	\item[Name]
	Usually self explanatory
	\item[Type ]
	The type of process that causes its explosion. There are a wide 
	variety : Propellant, Brisant, Gas, Electrochem, Fission, Fusion.
	\item[Damage (DM)]
	Each package has a base damage In the case of bulk explosive the value
	will be given in damage / unit mass. 
	\item[Damage Decrement (DD)]
	The decrease in damage per unit distance. This measure is based on 
	a situation with no atmosphere. Double the damage decrement for an 
	Earth normal atmosphere. 
\end{description}

\begin{table}[hb]
\centering
\caption{Explosive Affect Modifiers}
	\begin{tabular}{||l|l||} \hline
		Placed Charge  & DAM = 2.5* \\
		Shaped Charge  & DAM = 3.5* \\ \hline
	\end{tabular}
\end{table}



