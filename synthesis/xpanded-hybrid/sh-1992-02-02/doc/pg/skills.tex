\chapter{Skills}

A \ndx{skill} is an area of knowledge. A skill is defined as something that can
be learned and allows you to do something. Skills are used in almost every
action a player character performs and are discussed seperately from
other subjects because they are extremely important.

Proficiency in a skill is described by a number with a range of 0-20.
The higher the number, the greater the character's expertise. Someone
is completely unfamiliar with a skill is considered to be
\ndx{unranked}. Someone who is familiar with the basics of
the skill is rank 0. A talented amateur is has a rank 3 or 4. A
solid workaday craftsman has a rank 5-7. A professional is rank 8 or 9. 
An expert in a skill has a rank 12 - 14.
A true master of a skill has a rank 16-20 in a skill. 

\section{Skills}

\ndx{Skills} are described fairly simply:

\begin{relate}
	\item[Name] 
	Self Explanatory
	\item[Stat Basis] 
	The stat, or combination of stats, that is needed
	when using the skill.
	\item[Difficulty Factor] 
	The modifier for doing any action with this skill.
	\item[Cost] 
	The experience point cost is the amount of experience points it takes to
	buy a roll in a skill. 
	\item[Generation Cost]
	The cost in Skill Pool Points of a skill or skill package. Skills
	only cost 1 Skill Pool Point. Skill packages typically vary from 1
	to 10 Skill Points in cost.
\end{relate}

\subsection{Use}
\INDEX{Skill Use}{Skill Use}

To use a skill the GM determines what the Base Chance of the skill is 
and adds in the modifiers for the task being performed.

The Base Chance of using a skill is three times the Stat Basis of the skill
or \[ 3 \times SB_{skill} \] For each rank the character has in the skill add 4\%.
The modifiers for the task vary for each situation.

\example{
A character with rank:0 in Rock Throwing is throwing a rock 
across the street. The chaarcter has a Accuracy (ACC) of 12. Their Base Chance 
to hit is \( 3 \times 12 = 36\% \).
}

\subsection{Gaining}
\INDEX{Gaining Skills}{Gaining Skills}
A character can gain rank in skills by using a skill, gaining Experience Points
for its use and using those experience points to buy a learning roll for that
skill. 

The experience points gained for using a skill are:

\begin{relate}
	\item[Non-Ranked] Critical gains 50 experience. Normal Usage gains 30 exp.
	\item[Ranked] Critical gains 25. Normal usage gains 15.
\end{relate}

\subsection{Learning Rolls}
\INDEX{Learning Rolls}{Learning Rolls}

In order to gain rank in a skill the character uses experience points to
buy a learning roll in the skill. The experience point cost to buy a learning roll
is listed with the skill. Once the character spends
those experience points they roll to see if their rank in that
skill goes up.

The learning roll is based on the Success Chance of the skill with the  
DF for making the Learning Roll being: 

\[DF_{Learning\ Roll} = (-Rank) + ({Training\ Mods})\]

If the character fails to make the learning roll they gain a Df +2 to the next 
learning roll in that skill. The failure modifier is culmulative. A character
that has failed four times will get a DF +8 modifier to their next earning roll
in that skill.

\subsection{Training in Skills}

\ndx{Training} in a skill directly modifies the EF of the Learning roll in
that skill. Training Alone adds \[+.5 DF/10 hours\] 
Training under a Teacher adds \[+(.5 \times {Teachers\ Rank})/10 hours\]
Training with notes or study aids adds \[+(.5 \times {Teachers Rank}/3) \over {10 hours}\]

