\chapter{Computers}

\section{Outline}
\section{Description}

Computers are a bitch to model and require some care. In general computers
are modeled in the following manner. There are a number of assumptions made
including: Ram is cheap. Software and algorithms become fast enogh to make
multi gigabyte searches manageable. Storage Space is cheap . Software is not
cheap.

\section{Computer Description}


\begin{figure}[htb]
\caption{Computer Description Format}
	\begin{description}
		\item[Name]
		\item[Interface Type] (Visual-Mental-Aural-Tactile)
		\item[Interface Ability]
        \item[Interface Rank]
		\item[Processes]
		\item[IO]
		\item[Retrieval]
		\item[Storage]
	\end{description}
\end{figure}

\section{Computer Description Format Explanation}

\begin{description}
	\item[Name]
	Usually the most common name and the model specification.
	\item[Interface Type]
	(Visual|Mental|Aural|Tactile)
	\begin{description}
		\item[Visual]
		Light Board, Tele\-type, CRT, Holog\-ra\-phy, Kini\-esthi\-olotic Scanner
		\item[Mental]
		Droud Plug, Induction Helmet, Empathy Board, Psicoder
		\item[Aural]
		Vocoder, Sonic Interference Induction.,
		\item[Tactile]
		Feelie Board, Nerve Induction Board
	\end{description}
	\item[Interface Ability]
	\begin{description}
		\item[Abstract Language]
		\item[Native Language]
		\item[Direct Connect]
	\end{description}
    \item[Interface Rank]
    The effective rank that the interface maintains as a "Hardware 
	Filter Skill" . Basically a measure of how much of a characters 
	skills may be effectively transferred to the computer environment.
        This rating operates as an additive modifier to the character's 
        Comp Ops skill.
	\item[Processes (PRC)]
	The combined "pool" denoting the number and complexity of the processes
	that can be performed by the CPU.  Usually expressed in terms of the DF
	* Number of processes. Such that a 10 Processes * 6 DF/Process => PRC =
	60. Thus someone asks a EF 4 question. This is a DF = 6 so the computers
	rating is now 54.
	\item[IO]
	The rating denoting the combined number and size of information
	transfers that ca be performeds. Usually not of consequence except when
	dealing with large data transfers. If it does become of consequence , it simply subtracts from the EF of
	the process involved.
	\item[Retrieval]
	Discusses the DF * Number of Retrivals that the CPU can handle easily.
	DFs of retrievals are based on the query and the size of the Database
	being searched.
	\item[Storage]
	Discusses the Permanant storage area. Usually rated in terms of gigabytes.
\end{description}


\section{Artificial Intelligence}

A form of software that allows the CPU to partially interpret what the
user has asked. In effect, the CPU learns something about the user and
adds in various additional parameters to the query as needed. As a
result AI is a direct aid in the Formulation Of Questions and Queries.

\section{Asking Questions of a computer}

The types of queries that are used are the same as for the chapter on 
Research.

\subsection{Retrieval Queries}

Directly get some information. The easiest type of Question. EF = 10.

\subsection{Calculational Queries}

Given information, perform numeric manipulations on them.

\subsection{Inference Queries}

Given information, infer something from them. This is difficult but
fairly straight forward.

\subsection{Formulate Queries}

Given the following information what questions do I need to ask to get
the following. Rarely done and somewhat complex.

There are three steps to asking a detailed question

\subsection{Formulate}

In most situations this is done by the PC on an ad hoc basis and
treated on the fly by the GM. Usually the query is formulated as a
request to retrieve data. Only in special circumstances should the GM
require the character to roll in order to Formulate the query.


\subsection{Retrieve}

Retrieval is a direct search in the appropriate database of
information or related DB's.

\subsection{Calculate}
\subsection{Correlate}

\subsection{Examples}

The PC asks the question. If the interface is natural language the
character simply asks a question. This is a base EF of 5. For each
complexity, subtract from the EF until we have the chance of asking the
question appropriately. AI software will add to the EF and a complex
interface will subtract from this. The final roll ius made , taking into
account the rank in using the computer and the SN is determined. The S
of this roll is added to the SN of the actual query result.

The EF is purely the result of the combined EFs involved in the query.

\begin{table}[htb]
\caption{Typical EFS involved}
	\begin{tabular}{||l|l||} \hline
	Simple Database search      & 5 \\
	Simple Calculation          & 5 \\
	Simple Inference            & 4 \\
	Simple Formulation          & 3 \\ \hline

	Additional Factor Same type & -1 \\
	Additional Factor Different Type & -2 \\
	Database is larger than retrieval rating  & (-1/Magnitude) \\
	Database is smaller than retrieval rating & (+1/magnitude) \\
	Software is specific to problem           & (EF +3) \\
	Software is specific to type of problem   & (EF +2) \\
	IO is greater than IO rating              & (-1/Magnitude) \\ \hline

	AI Interface software                     & EF +1-+4 to formulation \\ \hline
	\end{tabular}
\end{table}

\subsection{Query 1}
The query involves a Database search (EF = 5). It is a small
database ( 7 Megabytes ). It is well below the retrieval rating of
the system. So that adds 2 to the EF. The database is specific to
what is being looked for ( EF + 2) .The end result is a EF 9 skill
roll. Most high tech level pocket calculators can handle this one.
Database search (EF = 5). A small database ( EF + 2). Add in a
simple calculation (EF -2), add in a correalation (EF -2) and add in
an inference (EF -2).

Again, these rules should rarely be needed


