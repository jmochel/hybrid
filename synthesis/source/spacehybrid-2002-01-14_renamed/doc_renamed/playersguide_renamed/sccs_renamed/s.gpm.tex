h39205
s 00383/00000/00000
d D 1.1 99/12/02 15:46:08 jmochel 2 1
cC
cK01151
cO-rw-rw-rw-
e
s 00000/00000/00000
d D 1.0 99/12/02 15:46:04 jmochel 1 0
c BitKeeper file G:/SpaceHybrid/Doc/PlayersGuide/gpm.tex
cBjmochel@devilmountain.bedford.foliage.com|ChangeSet|19991202203126|52994|e2968a7f5cb68f67
cHdevilmountain.bedford.foliage.com
cK44916
cPDoc/PlayersGuide/gpm.tex
cReb4785455cb68f68
cZ-05:00
c______________________________________________________________________
e
u
U
f e 0
f x 33
t
T
I 2
\chapter{General Play}

This chapter discusses various pieces of the game system that effect 
every character. These rules are not specific to either combat or 
non-combat situations.  

The model of tasks and actions in \SH\ is based on a series of 
reactions and actions. When a character first enters a scene
they determine how much they see and understand of the situation {\em
\ndx{perception roll}}. Then they determine how quickly they can react {\em 
\ndx{initiative roll}}. The character will react faster when they know what
\index{Initiative}
is going on and slower when they don't. If a character is expecting 
something to happen they can prepare for that occurrence {\em preset 
reactions} and speed up their response. 

Once the character has reacted they determine what they will do and then 
do the action. 

There are a variety of things that can modify the chance of doing an 
action successfully. The character can mentally prepare for the 
action {\em set-up} to increase their chances. The action can be sped 
up by decreasing the chance of success {\em rushing an action}. 
Actions can be performed simultaneously {\em floretine}. There are 
additional modifiers for doing something while moving or while tired 
and so on\dots

\section{Time Scale}

\index{Time Scale}
Time is referred to by the units we are used to, Hours, minutes, and seconds 
and one that is new: Pulses. A \ndx{pulse} is 1/10th of a second. Pulses are
used in combat and other time critical activities.

\section{Perception}

\index{Perception}
Most of the time a situation is self evident. A character automatically
knows that there is a bar in the room and how many people are in it.
But if something could go unnoticed by the character, such as a suprise 
attack or something hidden, the player should make a \ndx{perception roll}. 
A perception roll is typically SB = PAW, DF=0, with modifiers for 
how alert the character is trying to be. A perception roll takes
8 pulses. A \ndx{Passive Perception Roll} can be made
\index{Perception!Passive}
during any action at 1/4 the success chance of a normal perception roll. A
passive perception roll takes no time and takes no modifiers for 
simultaneous actions.

The critical success and failure effects are fairly straight forward. 

\begin{description}
	\item Amazing Success
        Total Understanding, 300% Detail, +-0% Timing

	\item Very Notable Success
        Total Identification, 200% Detail, +-5% Timing

    \item Notable Success
        Total Identification, 150% Detail, +-10% Timing

	\item Solid Success
        Able to Identify exactly what is happening, 125% Detail, +-25% Timing

	\item Success
        Basic Identification, 100% Detail, +-50% timing

	\item Failure
        Vague Identification, 25% Detail, +-75% timing

	\item Solid Failure
        No real clue, 0 Detail, 0 Timing

	\item Notable Failure
        Inaccurate Identification, +-125% Detail, +-175% Timing

	\item Very Notable Failure
        Inaccurate Identification, +-150% Detail, +-200% Timing

	\item Amazing Failure
        Wildly Inaccurate Identification, +-250% Detail, +-300% Timing
\end{description}

\begin{verbatim}
Task: Active Physical Perception  
DF: 0 
Time: 8 cts. 
Skills: General Perception, Combat Perception
Notes: 
\end{verbatim}

\begin{verbatim}
Task: Passive Physical Perception  
DF: 0 
Time: 0 cts. 
Skills: General Perception, Combat Perception
Notes: Done at 1/4 the normal chance 
\end{verbatim}

% FILE Initiative Roll Modifiers
% REF tab:UnengagedInitiativeMods
\begin{stable}{Perception Modifiers}{ll}
\label{tab:PM}
	Situation				&  DF \\ 
\TableSubtitleRule
	Blinded				 &  -5 \\
	Deafened				 &  -3 \\
	Drunk/Stoned			 &  -5 \\
	Asleep					 &  -4 \\
	Poor Lighting			 &  -3 \\
	Not Alert 				&  -3 \\
	Alert					&  0  \\
	Actively Watching		& +3  \\ 
\end{stable}


\section{\ndx{Initiative} (Who Goes First)}

When a character first enters a situation where action may be required they must determine
how much they know and how quickly they react.

When a character first becomes involved in a conflict they roll a perception roll. Then the PC rolls an initiative roll. The 
Initiative roll is simply $ 2d6 + 8 - Speed_{Reaction} $ added together. There are modifiers 

$ Speed_{Reaction} = 1/2 {Character's\ Speed } $

\index{Speed!Reaction}
\[{Initiative} = 2d6 + 8 - Speed_{Reaction}\]

If the perception roll is unsuccessful, the character adds a modifier 
to the roll. 

\[{Initiative} = 2d6 + 8 - Speed_{Reaction} + 5\]

There are, of course, modifiers to the perception roll as detailed in table 
~\ref{Table:PerceptionModifiers}

If the initiative roll is lower than 1 the excess speed goes toward speed points 
and can be applied to a number of seperate tasks.

\subsection{Speed Gains Due to Rank in a Skill}

The character may add \( Rank/2 \) points to their speed points when using a
skill. This may only be done once the character has decided to use a
\index{Speed!Gains from Skill Rank}
given skill. 

\subsection{How to avoid the math}

There is a set of tables that the GM can provide that can be used to
simplify this. 

\subsection{\ndx{Preset Reactions}}

When a character is waiting for something specific to happen and intends to react a certain 
way when it does the charcetr is presetting an action. A gunfighter waiting for someone else 
to start drawing their weapon is a preset action. Having a preset action allows the character to increase
the chance of detecting the triggering action and speeds up the preset action. 
Holding a preset action can be fatiguing over long periods of time.

Declaring an action to be preset allows an DF +4 to a perception roll. If the perception roll is successful, the 
character gets to apply their $ 2 \times Speed_{Reaction} $. A Preset reaction may only be held for MST 
in the time scale that the players are working in before a cost of 1 MFT must be expended.

\section{\ndx{Actions}}

Actions normally begin at the count given by the initiative roll. The 
must be made at this point. The speed of the action is determined and 
the character takes this action on a pulse given by Initiative + The speed of the action.  

\section{Speeds of Actions}

Most actions have a speed associated with them. All simple actions
\index{Speed!of Actions}
, unless otherwise noted, have a standard speed of 10 pulse. 


% Speeds of Basic Actions

% Speeds of Basic Actions
\begin{table}[hb]
\centering
\caption{Speeds of Basic Actions}
	\begin{tabular}{||l|l||} \hline
    Action						& Speed \\ \hline
	Lift Light object			&	5 \\
	Lift Heavy Object           &  10 \\
	Any Simple Physical Action  &   5 \\
	Perception					&	5 \\ \hline
	\end{tabular}
\end{table}

\index{Actions!speed of}
Actions can be performed faster. Speeding up an action lowers the
chance of success. Generally half the time to act means you have half the chance to succeed.

For each percentage of time units the action is sped up a corresponding percentage is removed from 
the success chance. Thus an action performed in 1/4 the time has 1/4 the success chance.

Actions can be sped up using speed points. 


\section{Drawing a Tool or Weapon}


This most often applies to drawing a weapon but can also apply to 
other tools. In general, when a weapon is in hand, all normal weapon 
speeds apply. In order to get a weapon into ones hand it takes 
\( 2 \times Speed_{weapon} \) in pulses. 

In order to get a weapon in hand faster than \( 2 \times 
Speed_{weapon} \) requires a fast draw roll against the weapon's 
skill. A successful ready roll brings the tool or weapon to bear at
\index{Speed!of drawing a weapon}
\index{Speed!of draw}
\( Speed_{weapon} \).


\section{Resolving an Action}

Actions usually require only a skill roll to be made. 

\subsection{Results}

The end result of a roll is either a numeric value or a simple 
subjective result. See the table ~\ref{Table:CriticalSuccess}

\section{Modifiers} 

This section describes the common Difficulty Factors for various 
situations. All these modifiers are culmulative.

\subsection{Unranked in the Skill}

A character who has training or experience in an action is
unranked in the required skill. Any character performing an action using 
a skill they have no rank in has 1/2 the normal base chance and adds 2 pulses to 
their initiative. 

\subsection{\ndx{Set-up}}

Waiting and prepping oneself for a task is called Setting-Up. It will generally increase 
the chance to do something at a cost of increased time to get it done.

Setting Up for an action takes as long as it takes to perform the action. The end effect is a
bonus to the Success Chance of \[ 20\% + 2\%/rank\]. 

To set-up an action with a time scale of pulses or seconds (and sometime minutes) the total time taken is \( 1 \times Speed_{Action} \).
To set-up an action with a time scale of many minutes, hours, days, or weeks the total time taken is \( 1/4 \times Speed_{Action} \).  

\subsection{Florentine}

Performing two actions at the same time is called Florentining.
\index{Florentine}
\index{Simultaneous Actions}
\index{Actions!Simultaneous}

One of the two actions is the primary action and it recieves a 
DF -3. The secondary action recieves a DF -6. This is only true 
if the two actions are both physical or both mental. If one 
action is a mental action and the other is a physical one the modifiers
go down to DF -2 and DF -4.

% Include General Modifiers Table
\begin{table}[hb]
\caption{General Modifiers}
\centering
	\begin{tabular}{||l||l|l||} \hline
	Situation			& EF Modifier	& Other Modifier \\ \hline
	No rank in skill 	& -4			& 				 \\
	Setup				& +2			& 2*RB			 \\ \hline
	\end{tabular}
\end{table}


\subsection{Physically Injured or Tired}

A character that is injured or fatigued has DF modifiers to 
their actions. Physical fatigue has the greatest effect on physical actions 
and Mental fatigue has the greatest effect on Mental actions.
\index{Injury}

\begin{table}[hb]
\caption{Physical Condition Modifiers}
\centering
	\begin{tabular}{||l|l||} \hline
	Situation			& EF Modifier	\\ \hline
	Out of PFT			& -2 physical	\\
	Out of PFT			& -1 mental		\\
	25\% wounded in PBD & -2 Physical	\\
	25\% Wounded in PBD & -1 Mental		\\
	50\% Wounded in PBD & -4 Physical	\\
	50\% Wounded in PBD & -2 Mental		\\ \hline
	\end{tabular}
\end{table}

% FILE Mental Condition Modifiers Table
% REF 

\begin{stable}{Mental Condition Modifiers}{ll}
	Situation			& DF Modifier	\\
\TableSubtitleRule
	Out of MEX 			& -6 mental 	\\
	Out of MEX			& -3 physical	\\
	25\% wounded in MBD & -2 Mental		\\
	25\% wounded in MBD	& -1 Physical 	\\
	50\% wounded in MBD & -4 Mental		\\
	50\% wounded in MBD & -2 Physical	\\
\end{stable}


\subsection{\ndx{Movement}}

When performing an action the character may be affected by 
his rate of movement. If the character is moving faster than a 
walk the DF due to movement applys to any physical action they are 
attempting. ANy mental action they perform is subject to 1/2 
the listed DFs.


\begin{table}[h]
	\begin{tabular}{ll}
	Double Move: jog							  & DF -2 \\
	Triple Move: Run							  & DF -4 \\
	Fast move: Dash							 & DF -6\\
	Vehicular Movement							& DF -10 \\ 
    \end{tabular}
    \caption{Movement Modifiers}
\end{table}

\subsection{Environmental Conditions}

This is a catchall area. Characters generally are at their best 
performance in conditions similiar to the environment in which they 
were raised. Any drastic modifications from that environment in 
terms of light, gravity, humidity, etc\dots can lower the character's 
performance.

% FILE Environmental Condition Modifiers
% REF 
\begin{table}[h]
	\begin{tabular}{ll}

	Situation			& DF Modifier	\\
	\hline
	Lighting 50\% off	& -3			\\
	Lighting 75\% off	& -4			\\
	Gravity 50\% off	& -3			\\		   
	Gravity 100\% off	& -4			\\

    \end{tabular}
	\caption{Environmental Condition Modifiers}
\end{table}



\section{\ndx{Fatigue and Exhaustion}}
\index{Fatigue}
\index{Exhaustion}

A character using energy to perform actions draws from two different 
types of reservoirs: Fatigue and Exhaustion. For physical actions the 
stats are Physical Fatigue and Physical Exhaustion (PFT and PEX). For mental
actions the stats are Mental Fatigue and Mental Exhaustion (MFT and MEX).

Fatigue is the quick access pool of energy a character can use.
Exhaustion is the reserve pool of energy a character can use. 

\subsection{Losing Fatigue}

A character loses fatigue as the result of physical activity or combat. 
A character that has lost all their fatigue has no modifiers to their actions.
Fatigue will come back quickly. For each 10 points of fatigue used the character 
also loses 1 point of exhaustion.

\subsection{Losing Exhaustion}
A character loses Exhaustion by performing strenous activity or by losing 
fatigue. There are modifiers for being low in Exhaustion.

% FILE PFT and PEX cost for a given activity
% REF 
\begin{table}{PFT and PEX costs for activity}
	\begin{tabular}{lcc}

	Activity		&	PFT & PEX \\	
\hline
	Crawling		&	   &	  \\
	Walking			& 1/min	& 6/hr \\
	Jogging			&		& 1/min \\
	Running			&	   & 6/min \\
	Dash			&		& 2/sec \\
	Chopping Wood	 & 3/min & 18/hr \\
    \end{tabular}
    \caption{PFT and PEX costs for activity}
\end{table}


Characters lose MFT and MEX in the same manner.

% FILE MFT Loss Numbers for a given activity
% REF 
\begin{table}[h]
\centering
\caption{MFT Loss Numbers for a given activity}
	\begin{tabular}{lr} \hline
	Activity		&	LN	\\ \hline	
	Psionic Combat	&	8	\\
	Training		&	6	\\  \hline
	\end{tabular}
\end{table}


\subsection{Restoring Fatigue and Exhaustion}

The restoration of Fatigue is usually very quick. Exhaustion and Fatigue
restore themselves independently of each other.

% FILE PFT and PEX gain
% REF 
\begin{SHTable}
	\begin{tabular}{lcc}
	Activity		&	PFT & PEX \\	
\hline
	Sitting/Talking	&	1/sec	& 2/hr \\
	Resting(prone)	&	1/sec	& 5/hr \\
	Sleeping		&	1/sec	& 10/hr \\
	Eating (Large Pasta like meal) 	 & 1/sec & 6 \\
    \end{tabular}
    \caption{PFT and PEX gains for activity}
\end{SHTable}


\section{Physical Movement}

Each character has a statistic named Physical Movement. This is the character's
movement in meters/second at a dash. There are a total of five different types
of movement that a character may utilize. Each type of movement has its own
movement rate which is derived from the character's movement statistic. 
Ideally the player will have the full range of movements listed on his 
character's sheet. 

% FILE Movement Types
% REF 
\begin{SHTable}
	\begin{tabular}{ll}
	Movement Type		& Rate of Movement (meter/second) \\ 
\hline
	No Move             & \( 0 * Movement \) \\
	crawls, slow walks 	& \( 0.50 * Walk  \) \\
	Walking             & \( 0.50 * Jog   \) \\
	Jog					& \( 0.50 * Run   \) \\
	Run                 & \( 0.50 * Dash  \) \\
	Dash                & \( 1.00 * Movement \) \\ \hline
	\end{tabular}
    \caption{Movement Types}
\end{SHTable}


If the movement is being resolved during a time scale of greater than every
pulse one can get the distance traveled by simply multiplying the movement 
of the individual times the time spent moving. The time spent accelerating
is ignored as being negligible.

\begin{quote}
Let us say that Joe Daring spends 15 seconds running down a deserted street.
If he doesn't run out of street he will have covered 4 * 15 = 60 meters. If 
this seems a bit short, keep in mind that a run is not a full dash. At a full
dash Joe would have covered twice the distance and would be slowing down pretty 
drastically due to losing wind.
\end{quote}

\subsection{Acceleration}

\marginpar{It is important to remember that the accelleration rules 
should only be used when the distance travelled by the characters 
over a {\bf short period} of time is important }

In dealing with movement on a pulse by pulse scale we need to actually
deal with acceleration. The sequence is quite simple. Whatever the
final movement  rate is that the character intends to use is considered
the target movement rate. When the character first starts moving he
makes an skill roll in order to start moving at the movement
rate just below the target movement rate. Once the roll is made the
character is now moving at that lower rate. On his next  initiative the
character may attempt to accelerate to the target movement. Note  that
the gain number is the movement rate. If an acceleration roll is failed
the end result is that the character drops to the next lowest available
movement  rate. 

\begin{quotation}
Reed Johnson has a movement of Dash 10, Run 5, Jog 2.5, Walk 1.3, Crawl .6
\end{quotation}

% FILE Targeted Action Movement Modifiers
% REF 
\begin{SHTable}
	\begin{tabular}{ll}
	Slow move: crawls, slow walks (combat )   & DF -2 \\
	Normal move: Walking					   & DF -4 \\
	Double Move: jog						   & DF -6 \\
	Triple Move: Run						   & DF -8 \\
	Fast move: Dash						   & DF -10\\
	Vehicular Movement						   & DF -14 \\ 
	\end{tabular}
    \caption{Targeted Action Movement Modifiers}
\end{SHTable}


\section{Mental Movement}
This is a measure of the character's speed of mental travel. It is 
usually only used in Psionics and Computer usage.

\section{Opposing Skill Rolls}

An opposing skill roll in a roll in which the character attempts to 
undo an action done previously by another character. Typically the SN 
of the original action is taken as a negative modifier to the current 
skill roll.

\section{Stealth and Concealment}

Opposing Skill Rolls

\section{Deception and Detection}

Opposing Skill rolls



E 2
I 1
E 1
