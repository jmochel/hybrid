h10613
s 00052/00000/00000
d D 1.1 99/12/02 15:44:14 jmochel 2 1
cC
cK38318
cO-rw-rw-rw-
e
s 00000/00000/00000
d D 1.0 99/12/02 15:44:11 jmochel 1 0
c BitKeeper file G:/SpaceHybrid/Doc/Paradigm/cartog.tex
cBjmochel@devilmountain.bedford.foliage.com|ChangeSet|19991202203126|52994|e2968a7f5cb68f67
cHdevilmountain.bedford.foliage.com
cK44905
cPDoc/Paradigm/cartog.tex
cReb4785255cb68f68
cZ-05:00
c______________________________________________________________________
e
u
U
f e 0
f x 33
t
T
I 2
\chapter{Cartography}

\section{Planetary Topography}

Typically planetary surveys and maps are concerned with three 
different things: Topography, Mineral, and Cultural points of interest.

\subsection{topography}
Most Topographic areas are delineated by colors with density of Vegetation 
indicated by crossthatch pattern.

\begin{itemize}
	\item[Cultivated Vegetation]
	Yellow
	\item[Uncultivated Vegetation]
	Light Green
	\item[Uncultivated Heavy Vegetation]
	Dark Green
	\item[Non-Vegetation Bearing Lithosphere]
	Orange
	\item[Hills]
	Light Brown
	\item[High Hills]
	Brown
	\item[Mountains]
	Dark Brown
	\item[Mixed Lithosphere/Hydrosphere]	
	Marsh Pattern
	\item[Flowing Hydrosphere]
	Blue with flow symbol
	\item[Solid Hydrosphere]
	White
	\item[Still, Shallow Hydrosphere]
	Light Blue
	\item[Still Hydrosphere]	
	Blue
	\item[Still, Deep Hydrosphere]
	Dark Blue
\end{itemize}

\subsection{Population and Cultural}

\begin{itemize}
	\item[Habitation]
	Circle with rough density of population indicator \( 10^0, 10^1, \dots \)
	\item[Capital]
	Circle with Cross in it.
	\item[Military]	
	???
	\item[Bridge]
	bowtie
\end{itemize}
E 2
I 1
E 1
