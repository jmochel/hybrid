%this file is an example of using postscript specials to give 
% headings a gray background  

\hsize=5.7in 
\font\rm=ptmr at 10pt
\font\norm=ptmr at 10pt
\font\tenrm=ptmr at 10pt
\font\twelverm=ptmr at 12pt 
\font\twelvebb=ptmb at 12pt
\font\fourteenbf=ptmb at 14pt

\nopagenumbers


\newdimen\grayht \newcount\grayheight
\def\getgrayheight#1#2#3{\grayht=#1bp \grayheight=#1  \setbox0=\vbox{#3}

\loop \ifdim\grayht < \ht0
    \advance \grayht by\baselineskip \advance \grayheight by #2 \repeat}

\def\graybar#1#2#3#4#5{\special{ps:gsave #1 setgray #2 #3 rmoveto #4 0 rlineto  0 -#5 rlineto
-#4 0 rlineto 0 #5 rlineto  closepath fill grestore}}
\def\grayhead{\graybar{.8}{-6}{13}{216}{\the\grayheight}}
\def\graybhead{\graybar{.7}{-28.8}{15}{12}{\the\grayheight}}
\def\graythead{\graybar{.8}{-6}{15}{380}{\the\grayheight}}
\def\grayahead{\graybar{.7}{-28.8}{13}{12}{\the\grayheight}}

\def\heada#1{{\hsize=210bp\vskip 24pt plus6pt minus4pt
{\twelvebb\parindent=6pt \baselineskip=14bp
\getgrayheight{20}{14}{#1}\item{}\grayhead #1\par}
\vskip 24pt plus 6pt minus 4pt}}

then \heada{this is a grayyed heading} would have a gray background.
note however that many print shops can't use slicks with more than black
and white (the photographing is the rub). this isn't a problem with ur
laser of course but if the ps file is going to a linotronic then to a
print shop, one should have the linotronic print directly to film if
possible.

\bye
