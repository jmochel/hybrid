\chapter{Skill Design}

Before we used a simple description for skills. In designing skills we need
a more complex one.

\section{The Full Skill Description}
\small

\begin{relate}
	\item[Name] Self Explanatory
	\item[Type] The concept of type is detailed by a root word that describes the base
	type and modifiers that describes the level of other interaction. So
	typically a skill type would be Basic Type, Interaction, Implementation.
	A simple Ballroom Dancing skill would have a Skill Type of
	Art, Assisted, Non-Tool Based . See below for descriptions of the
	categories.
	\item[Stat Basis] The stat, or combination of stats, that is needed
	when using the skill.
	\item[Rank Bonus] The additional gain in the success chance per rank
	in the skill. This is usually 4\% but it can vary for package skills
	and so on.
	\item[Cost]	The experience point cost is the amount of experience points it takes to
	buy a roll in a skill. It is derived from the type of the skill.
	\item[Generation Cost]
	The cost in Skill Pool Points of a skill or skill package. Skills
	only cost 1 Skill Pool Point. Skill packages typically vary from 1
	to 10 Skill Points in cost.
	\item[Typical Difficulty Factors]
	This is just a list of tasks that characters might want to do with a
	given skill, and what the typical Difficulty Factor is for that task.
	\item[Basic Type]
	The overall term that described the very broad area of
	endeavour that the skill falls into. The basic type may be any of the
	following:
	\begin{relate}
		\item[Art]
		An activity that has as its aim an affect upon the aesthetic senses of its
		audience.
		\item[Craft]
		An activity that is intended to mix Art with the production of some
		utilitarian object or effect.
		\item[Science]
		A series of disciplines intended to increase a codified body of
		knowledge.
		\item[Engineering]
		Any area of endeavour that attempts to apply a codified body of
		knowledge to the production of a desired physical effect.
		\item[Technical Study]
		An area of endeavour that is focused upon creating and maintaining the
		end result of the corresponding engineering discipline
		\item[Physical Discipline]
		Any area of endeavour based on muscle memory training.
		\item[Mental Discipline]
		Any area of endeavour based on purely mental manipulations without
		reference ...\footnote{How in the world do I describe this one}
		\end{relate}
		\item[Interaction]
		This area describes the various modifications upon the broad themes
		described above. Interaction may be any of the following:
		\begin{relate}
			\item[Assisted]
			Requires the work of one or more additional entity.
			\item[Un-Assisted]
			Requires only one entity.
		\end{relate}
	\item[Implemetation]
	Implementation describes any tools that need to be utilized in order
	to perform an action using the skill. It is based on the most common
	usage of a skill. It can be any of the following.
	\begin{relate}
		\item[Complex Tool Based]
		Requires some additional tools of a moderately complex nature that have been
		made especially for the skill.
		\item[Simple Tool Based]
		Requires some additional tools other than the entities mind or body.
		Computers and calculators are good examples.
		\item[Non Tool Based]
		Requires no additional tools other than the entities mind or body.
	\end{relate}
	\item[Focus]
	Focus is only associated with the sciences and it refers to whether
	the character studies the full science or specializes in some sub
	section of that science.
	\begin{relate}
		\item[Non Directed]
		Intended to add to a general body of knowledge
		\item[Directed]
		Intended to add to a specific area of a body of knowledge.
	\end{relate}
\end{relate}

\normalsize

If a character wishes to learn a skill not supplied by the rule
system the GM must figure out the appropriate values for that skill.
Typically the GM will break the skill down into its basic type,
implementation, interaction, and focus. After doing so he will use
that breakdown to calculate its experience point cost.

The first step is to determine the skill's stat basis. In order to do
so the GM must decide what action comprises 50\% or more of the activity
the character undertakes when using that skill. Once he has determined
what that action is he must decide what stats are used by that activity.

In general each skill is measured against the tertiary stats first , then the
secondary and finally the primary. MSE and PSE are never used as SB's.
MVM based skills are self evident.

Most scientific or purely mental skills are FCS based.
Sex is PCA. Lovemaking is FCS.

For actions in which the character is non skilled the SB's are fairly
simple. For perception rolls use GAW or MAW or PAW. For analysis rolls
use MCA for entity manipulated situations. Manipulated means that there
was a deliberate attempt on the part of someone else to direct someones
perceptions of this area. Use FCS for non-entity manipulated
situations. For memory rolls roll MST.

If the most common action is a mental one but has a physical component then
the skill is not based on FCS,MCA, or MDF alone.

If the most common action is a physical skill but has a mental component then the
skill is not based on ACC,PCA, or PDF alone.

Skills that are not based on the tertiary stats are usually either MAW
or PAW. Any skills where the most common action is a perception roll has
to be based on one of the awareness stats. If the action could not be
performed at all if the entities primary physical sense were removed
then the skill is PAW based\footnote{Is the last really true ?}.

% Include Generic DFs for Actions
% Generic EFs for Actions
\begin{table}[hb]
\centering
\caption{Generic EFs for Actions}
	\begin{tabular}{||l|r||}	\hline
	Basic Identification of Actions Needed		  & 10	\\ \hline
	Judgement of Quality                          & 9	\\ \hline
	Basic Perception Roll w/in area of SKill      & 8	\\ \hline
	Basic Action ( makes up 60\% or more of the   & 7	\\  
	actions made by someone using this skill)     &      \\
	Anyone of basic compentence would know this   &      \\
	action well.                                  &      \\ \hline 
	\end{tabular}
\end{table}


\section{Experience point costs}

% Include Costs of Skill Components Tables
% FILE Costs of Skill Components
% REF 
\begin{SHTable}
	\begin{tabular}{lll}
    BasicType           & Cost          & DF \\
	\hline
	Mental Disciplines      &       3         & -4 \\
	Art                     &       3         & -2 \\
	Science                 &       7         & 0 \\
	Engineering             &       5         & 0 \\
	Crafts                  &       4         & -1 \\
	Technical Study         &       4         & 0 \\
	Physical Disciplines    &       2         & 0 \\
    \hline
    Interaction                    & Cost          & DF \\
	\hline
	Unassisted              &       0         & 0 \\
	Single Assisted         &       1         & -1 \\
	Multiple Assisted       &       2         & -2 \\    \hline
    Tool Use                       & Cost          & DF \\
	\hline
	Non-Tool Based          &       0         & 0 \\
	Simple Tool Based      &        1         & -1 \\
	Complex Tool Based      &       2         & -2 \\    \hline
	\end{tabular}
    \caption{Costs of Skill Components}\label{Table:SkillComponentCosts}
\end{SHTable}




Note that the Appropriate Tech Index that the skill was learned at
must be recorded.\footnote{ How to handle upgrading skills for new TI's ?}

\section{Filter Skills}

There is a category of skills which affects the use of other skills
in an environment they were not designed to be used in. These skills
are called filter skills. A Filter skill is any skill that can allow
for the full expression of other skills in an environment other than
that for which those skills were designed for.

Typical filter skills include the following: 0-g maneuver, Tech
Level Lore, Culture Lore, Horsemanship and other vehicular combat
skills, Armor Wearing, and Computer operations.

For situations in which the character is attempting to apply a skill
in a environment he is not familiar with and that skill {\em must }
interact with that environment, then the rank in the filter skill
becomes the upper limit on the effective rank of the skill being
used.

As an example, if someone has a horsemanship skill at rank 5, he or she
may use their archery skill up to rank 5 without making any rolls
against their horsemanship. If the character has a higher archery skill
and wants to bring it all to bear on a shot, they must roll against
their horsemanship first in order to get the full use of the archery
skill.

\section{Skill Packages}

What we have discussed up till now has been single skills.
Quite often though it may be better to offer skill packages.
A skill package is a collection of conceptually related
skills that have an experience point cost less than the total
cost of all the skills. All skills in a skill package may be
used normally with the exception that package skills have a lower
Rank Bonus and EF modifier.

\section{Experience Point Cost}

Each skill package has an overall Experience point cost that is
the result of adding the cost of the most expensive skill together
with the cost of all the other skills times some factor. The actual value
of this factor is used to name the type of skill package.

{\bf Quarter Experience}
\[ 1 \times Most\ Expensive\ Skill + 1/4 \times All\ Others \]
All skills within this package have a RB = 1\%, EF = -2

{\bf Half Experience }
\[ 1 \times Most\ Expensive\ Skill + 1/2 \times All\ Others \]
All skills within this package have a RB = 2\%, EF = -2

{\bf Three Quarter Experience}
\[ 1 \times Most\ Expensive\ Skill + 3/4 \times All\ Others \]
All skills within this package have a RB = 3\%, EF = -2

\section{Making A Skill Roll}

Skill rolls for skills within a package are made as normal with the
appropriate modifiers for the package skills.

\section{Modifiers to Using Skills}
All normal modifiers that apply to single skills apply to skills in a package.

\section{Learning Rolls}
The SB of the learning roll for a skill package is the least favorable
of all the SBs in the skills that make up the package. A skill package
suffers an additional DF -1 to Learning Rolls.

\section{Modal Packages}

Modal packages ( pronounced M\=o-dal ) are packages of Mental or Physical disciplines
that require concentration to maintain, leading to the exclusion of any action that
requires skills outside the package. Many of the martial arts styles are modal
packages.  Modal packages do not have the DF = -2 modifier that most packages do
but they require a skill roll to enter and exit the mode of concentration.

\section{Unfamiliar Tools}
All normal modifiers for unfamiliar tools apply to package skills.

\section{Gaining Skills}
Skill Packages may only be gained by training in the package from
someone who already has that skill package. \footnote{ Include rules for
actually designing skill packages. }

\section{Experience Points Gain from usage.}

All normal experience point gains from usage apply to skills in a skill
package with the exception that a character may never get eeps for
being non-ranked in a skill package\footnote{Put in notes on buying
out skills from a package}.

A typical skill package would be :

\begin{verbatim}
Aikido (3/4 package)

	Dodging In		  (SB=PAG)
	Dodging Away	  (SB=PAG)
	Grapple			 (SB=PCA)
	Balance Throw	 (SB=PCA)
	Joint Throw		 (SB=PCA)

	Cost ( 50 + 3/4(50+50+50+50) ) = 200
	Generation Cost = 1+3/4(1+1+1+1) = 4

	Skill PAckage SB: PAG|PCA
	RB=4%

\end{verbatim}

\section{Designing Skills}

If a character wishes to learn a skill not supplied by the rule
system the GM must figure out the appropriate values for that skill.
Typically the GM will break the skill down into its basic type,
implementation, interaction, and focus. After doing so he will use
that breakdown to calculate its experience point cost.

The first step is to determine the skill's stat basis. In order to do
so the GM must decide what action comprises 50\% or more of the activity
the character undertakes when using that skill. Once he has determined
what that action is he must decide what stats are used by that activity.

In general each skill is measured against the tertiary stats first , then the
secondary and finally the primary. MSE and PSE are never used as SB's.
MVM based skills are self evident.

Most scientific or purely mental skills are FCS based.
Lovemaking is PCA.

For actions in which the character is non skilled the SB's are fairly
simple. For perception rolls use GAW or MAW or PAW. For analysis rolls
use MCA for entity manipulated situations. Manipulated means that there
was a deliberate attempt on the part of someone else to direct someones
perceptions of this area. Use FCS for non-entity manipulated
situations. For memory rolls roll MST.

If the most common action is a mental one but has a physical component then
the skill is not based on FCS,MCA, or MDF alone.

If the most common action is a physical skill but has a mental component then the
skill is not based on ACC,PCA, or PDF alone.

Skills that are not based on the tertiary stats are usually either MAW
or PAW. Any skills where the most common action is a perception roll has
to be based on one of the awareness stats. If the action could not be
performed at all if the entities primary physical sense were removed
then the skill is PAW based\footnote{Is the last really true ?}.

% Include Generic DFs for Actions
% Generic EFs for Actions
\begin{table}[hb]
\centering
\caption{Generic EFs for Actions}
	\begin{tabular}{||l|r||}	\hline
	Basic Identification of Actions Needed		  & 10	\\ \hline
	Judgement of Quality                          & 9	\\ \hline
	Basic Perception Roll w/in area of SKill      & 8	\\ \hline
	Basic Action ( makes up 60\% or more of the   & 7	\\  
	actions made by someone using this skill)     &      \\
	Anyone of basic compentence would know this   &      \\
	action well.                                  &      \\ \hline 
	\end{tabular}
\end{table}


\subsection{Experience point costs}

% Include Costs of Skill Components Tables
% FILE Costs of Skill Components
% REF 
\begin{SHTable}
	\begin{tabular}{lll}
    BasicType           & Cost          & DF \\
	\hline
	Mental Disciplines      &       3         & -4 \\
	Art                     &       3         & -2 \\
	Science                 &       7         & 0 \\
	Engineering             &       5         & 0 \\
	Crafts                  &       4         & -1 \\
	Technical Study         &       4         & 0 \\
	Physical Disciplines    &       2         & 0 \\
    \hline
    Interaction                    & Cost          & DF \\
	\hline
	Unassisted              &       0         & 0 \\
	Single Assisted         &       1         & -1 \\
	Multiple Assisted       &       2         & -2 \\    \hline
    Tool Use                       & Cost          & DF \\
	\hline
	Non-Tool Based          &       0         & 0 \\
	Simple Tool Based      &        1         & -1 \\
	Complex Tool Based      &       2         & -2 \\    \hline
	\end{tabular}
    \caption{Costs of Skill Components}\label{Table:SkillComponentCosts}
\end{SHTable}




Note that the Appropriate Tech Index that the skill was learned at
must be recorded.\footnote{ How to handle upgrading skills for new TI's ?}

\section{Designing Skill Packages}

\subsection{Experience Point Cost}

Each skill package has an overall Experience point cost that is
the result of adding the cost of the most expensive skill together
with the cost of all the other skills times some factor. The actual value
of this factor is used to name the type of skill package.

{\bf Quarter Experience}
\[ 1 \times Most\ Expensive\ Skill + 1/4 \times All\ Others \]
All skills within this package have a RB = 1\%, EF = -2

{\bf Half Experience }
\[ 1 \times Most\ Expensive\ Skill + 1/2 \times All\ Others \]
All skills within this package have a RB = 2\%, EF = -2

{\bf Three Quarter Experience}
\[ 1 \times Most\ Expensive\ Skill + 3/4 \times All\ Others \]
All skills within this package have a RB = 3\%, EF = -2


