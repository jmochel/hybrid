\chapter{Character Generation}

A \ndx{Player Character} (Abbreviated: PC) is an imaginary individual with their
own physical and mental abilities, skills, history,  etc \dots Each 
character has physical and mental attributes that are quantified in 
10 primary statistics. 

This chapter discusses the method used to generate characters. The chapter 
includes all the tables needed and the information is presented in roughly 
the order needed to generate the character.

\section{Statistics}

In \SH\ the physical and mental attributes of a character 
are described by a series of numbers called \ndx{statistics} or stats. 
Physical Strength is a typical statistic. The higher a
statistic is, the better the character's chance to do actions using  
that stat. 

In generating the statistics of the character the player distributes 155
points among the 10 primary statistics. The minimum number of points that may be
put into a statistic is 5. The maximum amount that may be placed into
a statistic is 25. There are modifiers for the character's race and gender that 
are added on later. 

The next section gives the name and abbreviation of each statistic and 
describes what the statistic represents.

\newpage
\small

\subsection{Primary Statistics}
\begin{relate}
	\item[Physical Strength	(PST)]
	Physical Strerngth is the overall power of body. This represents 
	the character's raw muscle power and is not tied to some particular 
	set of limbs. Actions such as lifting are based on PST.
	\item[Physical Endurance (PEN)]
	Physical Endurance is the physical resilience and stamina of body. 
	This is a measure of the character's overall endurance and ability 
	to resist hardship as well as the ability to bounce back from hardship.
	\item[Dexterity	(DEX)]
	Dexterity is the eye and hand coordination and speed of hand movement. 
	This is	specifically tied to the character's manipulatory limbs. 
	\item[Physical Agility (PAG)]
	Physical Agility is the overall flexibity and responsiveness of body. 
	This affects the whole body actions of the character. 
	\item[Physical Awareness (PAW)]
	Physical Awareness is how sensitive the character is to 
	the physical part of the environment.
	\item[Mental Strength (MST)]
	Mental Strength is raw mental power. It is a measure of the characters overall
	computational and reasoning strength. It is also associated with 
	the character's strength of will. 
	\item[Mental Endurance (MEN)]
	Mental Endurance is resilience and stamina of mind. It is a measure of the
	mind's ability to recover from shock or disorientation. 
	\item[Mental Agility (MAG)]
	Mental Agility is the overall flexibility and responsiveness of mind.
	In another day and age this might be called ``Cunning'', ``Canniness'', or ``Shrewdness''.
	\item[Psi Potential (PSI)]
	Psi Potential is a measure of how easily a character can gain access to 
	their ``supernatural'' or psychic abilities.
	\item[Mental Awareness (MAW)]
	Mental Awareness is how sensitive the character is to 
	the non-physical part of the environment. 
\end{relate}

\begin{stable}{Primary Statistics}{lll}
	Stat & Overall  & Human \\ 
		 & Range	& Average \\ 
\TableSubtitleRule
	PST & 1-30		& 15 \\
	PEN & 1-30		& 15 \\
	DEX & 1-30		& 15 \\
	PAG & 1-30		& 15 \\ 
	MST & 1-30		& 15 \\
	MEN & 1-30		& 15 \\
	MAG & 1-30		& 15 \\
	PSI & 1-30		& 15 \\
\end{stable}

\newpage
\subsection{Calculated Statistics}

These statistics are determined from the primary stats. Like the
primary statistics they break down evenly into mental and physical
categories.

\begin{relate}
	\item[Physical Body (PBD)]
	The amount of physical damage a character can absorb. Derived from PST
	and PEN.
	\[(PST+PEN) \times RAC MOD \]
	\item[Physical Fatigue (PFT)]
	The amount of energy a character can expend, either in damage or in
	work. Derived from PEN and PAG.
	\[(2 \times PEN)+PAG\]
	\item[Physical Exhaustion (PEX)]
	The amount of energy reserve a character can expend as the result of 
	damage or from work. Derived from PEN and PAG.
	\[(4 \times PEN)+PAG\]
	\item[Physical Movement (PMV)]
	A measure of the character's movemnt. Derived from STR and END and
	racial modifiers.
	\[((PST+PAG)/5) \times RAC MOD\]
	\item[Mental Body (MBD)]
	The amount of mental damage a character can absorb. Derived from MEN and MST.
	\[(MST+MEN) \times RAC MOD\]
	\item[Mental Fatigue (MFT)]
	The amount of mental energy a character can expend, either in damage
	or in work. Derived from MEN and MAG.
	\[(2 \times MEN)+MAG\]
	\item[Mental Exhaustion (MEX)]
	The amount of mental reserve energy a character can expend, either in combat
	or in work. Derived from MEN and MAG.
	\[(4 \times MEN)+MAG\]
	\item[Mental Movement (MMV)]
	A measure of the characters rate of movement in the purely mental 
	realms of psionics and computer interfaces.
	\[(MST+MAG)/5)\times RAC MOD \]

	\item[Accuracy (ACC)]
	A measure of the character's effectiveness with projectile or missile
	weapons. Derived from STR and DEX. Could also be called Physical
	Accuracy.
	\[(PST+DEX)/2		  \]
	\item[Physical Combat Ability (PCA)]
	A measure of a character's ability to in\-flict 
	dam\-age in hand-to-hand and melee combat. Derived from 
	STR, PAG, DEX.
	\[(PST+DEX+PAG)/3 \]
	\item[Physical Defense (PDF)]
	A measure of a character's ability to dodge or evade  to avoid
	taking damage in hand-to-hand or melee combat. Derived 
	from PAG, DEX.
	\[(PAG+DEX)/2 \]
	\item[Focus	(FCS)]
	A measure of the character's effectiveness with focused mental
	actions. Derived from MST and MAG. could also be called Mental
	Accuracy.
	\[(MST+MAG)/2 \]
	\item[Mental Combat Ability (MCA)]
	A measure of a character's ability to inflict damage in Mind to Mind
	combat. Derived from MST, MAG, PSI
	\[(MST+PSI+MAG)/3 \]
	\item[Mental Defense (MDF)]
	A measure of a character's ability to avoid taking damage in mental
	combat or highly stressful situations. Derived from MAG.
	\[(MAG+PSI)/2 \]
	\item[General Awareness (GAW)]
	A measure of a character's connection and response to his Mental and
	physical senses. Derived from MAW and PAW.
	\[(MAW+PAW)/2 \]
\end{relate}

\normalsize
\onecolumn

% Include Statistic Calculation Tables
% FILE Tables of Primary and secondary statistics
% REF 

\begin{SHTable}[h]
	\begin{tabular}{l|l|l|l}
	Stat & Formula										& Typical & Human \\ 
		 &								   				& Range	 & Average \\ 
	\hline
	PBD  & \((PST+PEN) \times {Racial\ Modifier}\)	    & 02-60	 & 30  \\
	PEX  & \((4 \times PEN) + PAG \)		 	        & 15-150 & 75  \\
	PFT  & \((2 \times PEN) +PAG\)			& 09-90 & 45  \\
	PMV  & \(((PAG+PST)/5) \times {Racial\ Modifier}\) & 0.4-12 & 6 \\ 
	\hline
	MBD  & \((MST+MEN) \times {Racial\ Modifier}\)		& 02-60	 & 30 \\
	MEX  & \((4 \times MEN) + MAG \)		& 20-150 & 75  \\
	MFT  & \((2 \times MEN) + MAG \)		& 09-90	 & 45 \\
	MMV  & \(((MAG+MST)/5) \times {Racial\ Modifier}\) & 0.4-12  & 6 \\ 
	\hline
	ACC  & \((PST+DEX)/2	 \)				& 3-30  & 15 \\
	PCA  & \((PST+DEX+PAG)/3 \)				& 3-30  & 15 \\
	PDF  & \((PAG+DEX)/2	 \)				& 3-30  & 15 \\
		 &									&		&	 \\ \hline
	FCS  & \((MST+MAG)/2	 \)				& 3-30  & 15 \\
	MCA  & \((MST+PSI+MAG)/3 \)				& 3-30  & 15 \\
	MDF  & \((MAG+PSI)/2	 \)				& 3-30  & 15 \\
	\end{tabular}
\caption{Secondary Statistics Table}\label{Table:SecondaryStatistics}
\end{SHTable}

% EOF


\clearpage
\section{Other Statistics}
\begin{relate}
	\item[Race]
	Race describes the actual genetic background of the character. 
	\item[Gender]
	If Applicable.
	\item[Height (HT)]
	Expressed in Centimeters. If the creature being described is
	quadrapedal, the height given is the height to the shoulder.
	\item[Weight (WT)]
	Expressed in Kilograms.
	\item[Appearence	(APP)]
	A measure of how physically attractive a character is 
	to others of their race.
\end{relate}

\subsection{Character Speeds}

For each of the awareness stats there are two speed statistics. 
There are Unengaged Speed and Engaged Speed stats next to each
awareness stat. These are derived from the following table:

% FILE Initiative Roll Table
% REF tab:CS
\begin{table}[h]
	\begin{tabular}{ll}
	SB 		& Speed \\
	\hline
	1--2	& 0		   \\
	3--3	& 1		   \\
	4--5	& 1		   \\
	6--8	& 2		   \\
	9--11	& 3		   \\
	12--15  & 4		   \\
	16--19  & 5		   \\
	20--24  & 6		   \\
	25--29  & 7		   \\
	30--34  & 8		   \\
	35--35  & 9		   \\
	36--39  & 9		   \\ 
	40--45  & 10	   \\ 
    \end{tabular}
    \caption{Reaction Speed Table}
\end{table}


\subsection{Racial Modifiers}

The player must decide, with the GM's agreement, on the race that the 
character will be. Once that decision is made any racial modifiers 
must be applied to the statistics. From that point on, no statistic 
may be allowed to exceed the racial maxima.

\subsection{Gender Modifiers}

Some races may have modifiers to the statistics of the PC based on 
the gender of the character. 

\subsection{Cultural Modifiers}

Members of a specific cultures should only rarely have modifers 
to their statistics as a result of being a member of that culture.
Such modifiers would indicate a culture wide obsession with a given 
physical or mental type. 

A GM may have specific skills or skill packages specific to a culture.

\subsection{Determination of Height and Weight}

If the character has no preference for the height of the characterit 
may be randomly generated using the following formula: \\

\[ Height = {Average Height} + ( {{2d10 - 11} \over {10}} \times 
{Height Variation} )
\]

\[ Weight = {Average Weight} + ( {{2d10 - 11} \over {10}} \times 
{Weight Variation})
\]

These figures are based on an adult of the given race and the values
for these variables can be found in the description of the race.

\clearpage
\section{Background and Careers}

The next step is determining the background of the character
and the career path they took up until this time.

\subsection{Background}

The begining character starts out with a set of skills from either 
general life experience or formal educational training. As a result of 
this education the character gains 20 skill points. These skill points 
may be used to buy skills or skill packages. The types of skills that may
be gained must be consistent with the type of education the character 
recieved. 

The skills gained should be simple skills. !!!! Expand

As a result of just living in the environment they grew up in 
the character automatically gets skills in the lore of 
their own culture, skills in the language, and simple skills 
covering movement in the home environment.

These skills are listed in the table below.

\subsubsection{Initial Skill Points}
\begin{enumerate}
	\item 20 points of education skills.
	\item 20 points of skills in written and spoken language.
	\item 20 points of skill in Cultural Lore. Both local and larger scale.
	\item 20 points of skill in Lore:[Tech Index] of Culture.
	\item 15 points of manuevering skills for the native environment
\end{enumerate}

\subsubsection{Upper Limit to Skill Rank}

There is a limit to the highest rank in a skill a character can 
achieve during character generation. This limitation 
is based on the stat basis of the skill and is only applicable to to 
the basic education skills. 

A summary of that limitation is presented in table \dots.

% Include the Learning Limitations Table
% FILE Table of skill rank limit 
% REF

\begin{stable}{Limit in Skill Ranks from Base Education}{cl}
	Stat Basis & Rank (Upper Limit) \\
\TableSubtitleRule
	03-05		& 0 \\
	06			& 2 \\
	07			& 3 \\
	08			& 4 \\
	09-10		& 5 \\
	11-12		& 6 \\
	13-19		& 7 \\
	20-34		& 8 \\
	35+			& 9 \\
\end{stable}


Don't worry if many of the skill names and costs don't make sense 
yet, they will be explained.

\subsection{Careers}

This phase of character generation is where a majority of a character's
skills and history will be developed. 

In delineating a character's history it is assumed that he or she has
embarked on a course of career or non-career that brings him into
contact with chances to raise his skills, his status, and his material
wealth.

During the rolling of a characters career points gained are added 
and subtracted to or from four ``pools''. These four pools : Stats, 
Skills, Wealth, and Status are the basis of the final resolution of 
the character's skills, history, and station.

Each of the available paths has its own advantages and disadvantages.
Educational careers give one little chance to injure one's self but the
possible monetary gains are low.  Military careers are dangerous, but
possibly fairly rewarding.

The four pools each have a different basic function.

The Statistics pool or ``Stat'' pool, serves as the point pool for 
increasing a PC statistics or creating special abilities.

The Skill pool is the repository of points to be spent in gaining 
skills and skill packages.

The Wealth pool contains points to be spent in establishing the
character's basic financial state.

The Status pool contains points to be spent in gaining all the 
possible trappings of status: reputation, syncophants, or recognition.

Players do have a limited amount of lateral movement for these  points.

In addition a player can add and subtract to/from the pools by the 
usage of Advantages and Disadvantages.

Each career is delineated by a simple set of numbers combined with a
simple description. It has the following format:

\begin{description}
	\item [Name]
	Self explanatory
	\item[Base Skill Points]
	The base skill points is how many skill points are normally
	recieved during a one year period of endeavor.
	\item[Base Financial Gain]
	The base amount of stads(standards) gained in a one year period of a career.
	\item[DF (Stat|Skill|Wealth|Status)]
		\item[Stat]
		The DF for a GAW roll to determine the overall effect the career
		had during a given time period on an entities health. Success 
		puts points into a Stat Pool that can be used for raising primary
		statistics. Critical failure effects usually involves the loss of
		points from the Stat Pool.
		\item[Skill]
		The DF for a GAW roll to determine whether or not the character
		gained any skills during this time period. The GN = Base Skill 
		Points associated with the career and is applied to the Skill 
		Pool. Critical failure effects do not apply to this roll.
		\item[Wealth]
		The EF for a GAW roll to determine whether or not the character
		gained in the material or financial area. GN = Base Financial 
		Gain of the career. 
		\item[Status]
		The DF for a GAW roll to determine whether or not the character
		gained in the area of Status. Status is a rather subjective thing
		but typically, a military career leads to increases in rank and
		possible minor fame. A increase in status in a shadowy career would
		lead to the development of a "Rep". Obviously, the types of gain would
		need to be negotiated between the player and the GM. The GN is 1.
		\footnote{All careers should have a rank gain cost in status points}
\end{description}

{\bf Typical Careers}
	\begin{relate}
		\item[Pick Pocket]
		4rnks:4,000stads:(5-6-4-6) 
		\item[Smuggler]
		5rnks:10,000stads:(4-7-5-3)
		\item[Terran Space Navy]
		6rnks:14,000stads:(4-7-3-3)
		\item[Grunt Mercenary]
		5rnks:7,000stads:(4-7-3-5)
		\item[Scouts]
		6rnks:4,000stads:(4-8-4-5)
		\item[Nurse]
		8rnks:12,000stads:(6-7-5-3)
		\item[Traffic Controller]
		6rnks:25,000stads:(5-6-3-2)
		\item[Advanced Education]
		8rnks:2000stads:(6-7-3-3)
	\end{relate}

The example careers listed above would usually be fleshed out with 
additional detail such as a description of rank and status, etc\dots
\footnote{An important question is that of when a character is
allowed to drop out of military and so on careers.}

\section{Designers Notes}

The nominal maximum number of points that may be gained each
resolution period is as below. This is not the maximum that the 
character may roll , but it is the maximum that may be a part of the 
design of a career.

\begin{itemize}
	\item[Wealth] 10 points
	Wealth values are table based as per career ? Should this be ?
	\item[Status] 2 points
	\item[Skills] 10 points
	\item[Health] 2 points
\end{itemize}

\section{Buying Skills and Advantages}

In the previous portion of the character generation process we added 
and subtracted points to four ``pools''. These four pools : Stats, 
Skills, Wealth, and Status are the basis of the final resolution of 
the character's skills, history, etc \dots

The four pools each have a different basic function.

The Statistics pool or ``Stat'' pool, serves as the point pool for 
increasing a PC statistics or creating special abilities.

The Skill pool is the repository of points to be spent in gaining 
skills.

The Wealth pool contains points to be spent in gaining an idea of 
basic financial state.

The Status pool contains points to be spent in gaining all the 
possible trappings of status. Reputation, syncophants.

Players do have a limited amount of lateral movement for these 
points. The Wealth and Status pool can exchange to a maximum of ???
The Statistics and Skill pools can exchange a maximum of ???
points.

In addition a player can add and subtract to/from the pools by the 
usage of Advantages and Disadvantages.

\section{Skill Costs}

When characters are first generated their skills are purchased 
using points from the skill pool. Typically all skills cost 1 skill
point to purchase at the beginning and all skill packages cost more than
one point. 

Once characters have been generated all advancement and gain in skill
ranks is purchased with experience points.

\section{Skill Packages}

What we have discussed up till now has been single skills.
Quite often the character will be using a skill package.
A skill package is a collection of related 
skills that have a bundled experience point cost.
All skills in a skill package may be used normally with the 
exception that package skills have a lower Rank Bonus and DF.

A typical skill package would be :

\begin{verbatim}
Aikido (3/4 package)

	Dodging In		  (SB=PAG)
	Dodging Away	  (SB=PAG)
	Grapple			 (SB=PCA)
	Balance Throw	 (SB=PCA)
	Joint Throw		 (SB=PCA)

	Cost ( 50 + 3/4(50+50+50+50) ) = 200
	Generation Cost = 1+3/4(1+1+1+1) = 4

	Skill PAckage SB: PAG|PCA

\end{verbatim}

 

\section{Enhancements and Limitations}

After a character has some points in the four pools associated with 
character creation ( health, wealth, status, skill) he may choose to 
use them to buy Enhancements that will add flavour to the character. 

There are always the base enhancements allowed to the character. The
points in the Skill pool can be used to buy skills at the generation
cost. The points in the Stat pool can be used to buy stats at the costs
listed in the stat cost tables.

Enhancements are gains in either background or special abilities that 
can be paid for with points from one of the pools. There are two main 
types. There are special abilities or SPABS and there are 
Environmental Enhancements.  

SPABS are natural aptitudes that are typically permanent and inherent 
to the character's makeup.

Environmental Enhancements are typically advantages that depend on  the
character to maintain them. Such as wealth and various components of
status. 

% Read in the Cost tables 
% FILE Cost of Statistics
% REF 
\section{Statistics Costs}

\begin{table}[h]
	\begin{tabular}{llllll}
    Stat & Cost	&	Stat& Cost	& Stat	& Cost \\
\hline
	PST  & 4	&	PBD	&	2	&	ACC	&	2	\\
	PEN  & 3 	&	PSE	&	5	&	PCA	&	3	\\
	DEX  & 3	&	PFT	&	1	&	PDF	&	2	\\
	PAG  & 5	&		&		&		&		\\ 
	PAW	 & 4	&	PMV	&	4	&		&		\\
\hline	
	MST  & 8	&	MBD	&	2	&	FCS	&	2	\\
	MEN  & 6	&	MSE	&	5	&	MCA	&	3	\\
	MAG  & 12	&	MFT	&	2	&	MDF	&	2	\\
	PSI  & 10	&	MMV	&	4	&		&		\\ 
	MAW	 & 4	&		&		&		&		\\ 
	\end{tabular}
    \caption{Cost of Statistics}\label{Table:StatCosts}
\end{table}

% EOF



\section{Special Abilities}

SPABS have both Depth and Scope to help govern their cost.

Depth refers to the numeric advantage given by the SPAB in a given 
area. The scope denotes the number of different areas that the spab 
may be applicable to. 

In the case of raising a character statistic the scope refers to the 
number of other stats affected.

A SPAB that involves enhancing a statistic is different from raising the
statistic. A raised statistic ends up increasing the SB of the character
in that stat. The enhanced stat SPAB increases the Rank of the character
for any direct rolls against that stat. Thus an Enhanced Stat SPAB
affects Saving Throws and Concentration Checks.

\subsection{Enhanced Stats}
SPABs that involve stats have a cost identical to the cost of raising
the stat with the following exceptions.

Physical Awareness (PAW) has a scope of 5, Hearing, Feeling, Tasting, 
Seeing, Smelling. This means that to get a SPAB: Enhanced Physical
Awareness costs \( 5 \times 2 \) or 10 points.

MAW has a scope of one. 

GAW has a scope of 6. 5 PAW and 1 MAW.

\subsection{Ambidexterity}
\subsection{Eidectic Memory}
\subsection{Presence}
\subsection{Lightning Calculator}
\subsection{Mage Ability ?}
\subsection{Other SPAB Costs}

Most other SPABs have effects that can be linked to one of the
statistics. 

\section{Environmental Enhancements}

Environmental Enhancements have both Depth and Scope to help govern 
their cost. 

\subsection{Wealth}

The first and most commonly used is the wealth advantage. This differs
from the basic wealth that can be gained by spending the points from the
wealth pool in that the gain is approximately one third that of a pure
monetary spend, but the gain so obtained is income that will continue to
be generated for as long as the PC pays attention to the interests that
generate the funds. The larger the income the more work involved in
maintaining it. 

\subsection{Friends and Allies}

Another important Environmental Enhancement is that of Friends. Friends
have a depth associated with them dependent on how much they can be
counted upon. The scope is dependent on how easily the PC can access
that Friend. 

\begin{stable}{Friend and Ally Costs}{lll}
	Value	&	Dependability & Scope \\ \hline
	1		&	10 \% & 10 \%	\\
	2		&	20 \% & 20 \%	\\
	3		&	30 \% & 30 \%	\\
	4		&	40 \% & 40 \%	\\
	5		&	50 \% & 50 \%	\\
	6		&	60 \% & 60 \%	\\
	7		&	70 \% & 70 \%	\\
	8		&	80 \% & 80 \%	\\
	9		&	90 \% & 90 \%	\\
	10		&	100 \% & 100 \%	\\ \hline
\end{stable}

\subsection{Reputation}

\subsection*{Continued Careers}
A continued career as a Law Enforcement Officer or soldier is a 
balanced ad and disad situation.

\section{A Checklist}

To create a Player Character, you start with 

\begin{itemize}
	\item Initial Point Allocation
		\begin{itemize}
			\item Allocate Primary Statistics
			\item Calculate Secondary Statistics
			\item Calculate Tertiary Statistics
		\end{itemize}
	\item Education
		\begin{itemize}
			\item Pick basic skills derived from education
		\end{itemize}
	\item Background Generation
		\begin{itemize}
			\item Pick a career or a set of careers
			\item Pick the skills gained from those careers
			\item Rearrange Point Pool Gains
		\end{itemize}
	\item Limits and Enhancements
		\begin{itemize}
			\item Pick Limitations and Enhancements
		\end{itemize}
\end{itemize}
	


