\chapter{General Playing Mechanics}

\section{Introduction}

This chapter discusses various critical pieces of the game system that 
effect just about every character. They are not specific to either 
combat or non-combat situations. 

In situations that involve any type of conflict, whether physical, 
mental or verbal, the decision on who acts first may be 
critical\footnote{ Note in the design notes that we have intentionally 
removed first action determination from the Combat section}.

The total model of \SH centers around an reaction/action sequence.
The character determines when his reaction will occur and at that time 
declares his action. The action speed is added to the time of the reaction 
and the total is when the actual action finishes. Once the action has 
occured the character rerolls initiative unless he is acting on a
preset action. 

\section{Time Scale}

Time is typically broken down into the following common units:

% Include Time Scale Table
% FILE Time Scale Table
% REF 
\begin{SHTable}
\begin{tabular}{ll} 
	Pulse      & 1/10 Second	\\
	\end{tabular}
	\caption{Time Scale}	
\end{SHTable}


In general the more tense or critical an action is, the smaller the unit
of time that is used by the GM. 

\section{Initiative or Reaction}

Each entity that is involved in a conflict of any type must roll an 
EF -3 (not actively alert), EF +0 (Alert),  or EF +2 (Actively 
expecting trouble) initiative roll. Using the Initiative Roll Table 
the character uses the SB and the EF to get the value that is added to a
d10 roll and that is the time (generally in counts) it takes before a
character can react. Characters can specify that they wish to utilize 
a specific form of Awareness such as PAW. If they do so the add an 
additional EF +2 to rolls requiring that Stat but get an EF -4 
modifier to all other rolls utilizing the complementary stats. 

All initiative rolls are also open ended upon a roll of 10 or 1.
\footnote{The Formula for initiative is quite simple : \(Init= 
Roll_d10 + ( 15 - {(SB*EF) over 10}\) }\footnote{ Should this be an 
open ended roll}

% Initiative Roll Table
% Initiative Roll Table
\begin{table}[htb]
\centering
\caption{Initiative Roll Table}
	\begin{tabular}{||l|l|l|l|l|l|l|l|l|l|l|l|l|l|l|l|l|l|l|l|l|l|l|l|l||} \hline
		 & -10&	 -9	& -8 &  -7	& -6 &-5  &-4 & -3	&-2	 &-1   &0  &1   &  2 & 3  & 4 &  5 &  6 &  7  &  8 &  9 & 10 & 11 & 12 & 13 \\
       3 &  18&   17&  17&   17 & 16 & 16 & 16&  15 & 15 & 15  &15 & 14 & 14 & 14 & 13&  13&  13&  12 & 12 & 12 & 12 & 11 & 11 & 11 \\
       4 &  19&   18&  18&   17 & 17 & 17 & 16&  16 & 15 & 15  &15 & 14 & 14 & 13 & 13&  13&  12&  12 & 11 & 11 & 11 & 10 & 10 &  9 \\
       5 &  20&   19&  19&   18 & 18 & 17 & 17&  16 & 16 & 15  &15 & 14 & 14 & 13 & 13&  12&  12&  11 & 11 & 10 & 10 &  9 &  9 &  8 \\
       6 &  21&   20&  19&   19 & 18 & 18 & 17&  16 & 16 & 15  &15 & 14 & 13 & 13 & 12&  12&  11&  10 & 10 &  9 &  9 &  8 &  7 &  7 \\
       7 &  22&   21&  20&   19 & 19 & 18 & 17&  17 & 16 & 15  &15 & 14 & 13 & 12 & 12&  11&  10&  10 &  9 &  8 &  8 &  7 &  6 &  5 \\
       8 &  23&   22&  21&   20 & 19 & 19 & 18&  17 & 16 & 15  &15 & 14 & 13 & 12 & 11&  11&  10&   9 &  8 &  7 &  7 &  6 &  5 &  4 \\
       9 &  24&   23&  22&   21 & 20 & 19 & 18&  17 & 16 & 15  &15 & 14 & 13 & 12 & 11&  10&   9&   8 &  7 &  6 &  6 &  5 &  4 &  3 \\
      10 &  25&   24&  23&   22 & 21 & 20 & 19&  18 & 17 & 16  &15 & 14 & 13 & 12 & 11&  10&   9&   8 &  7 &  6 &  5 &  4 &  3 &  2 \\
      11 &  26&   24&  23&   22 & 21 & 20 & 19&  18 & 17 & 16  &15 & 13 & 12 & 11 & 10&   9&   8&   7 &  6 &  5 &  4 &  2 &  1 &  0 \\
      12 &  27&   25&  24&   23 & 22 & 21 & 19&  18 & 17 & 16  &15 & 13 & 12 & 11 & 10&   9&   7&   6 &  5 &  4 &  3 &  1 &  0 & -0 \\
      13 &  28&   26&  25&   24 & 22 & 21 & 20&  18 & 17 & 16  &15 & 13 & 12 & 11 &  9&   8&   7&   5 &  4 &  3 &  2 &  0 & -0 & -1 \\
      14 &  29&   27&  26&   24 & 23 & 22 & 20&  19 & 17 & 16  &15 & 13 & 12 & 10 &  9&   8&   6&   5 &  3 &  2 &  1 & -0 & -1 & -3 \\
      15 &  30&   28&  27&   25 & 24 & 22 & 21&  19 & 18 & 16  &15 & 13 & 12 & 10 &  9&   7&   6&   4 &  3 &  1 &  0 & -1 & -3 & -4 \\
      16 &  31&   29&  27&   26 & 24 & 23 & 21&  19 & 18 & 16  &15 & 13 & 11 & 10 &  8&   7&   5&   3 &  2 &  0 & -1 & -2 & -4 & -5 \\
      17 &  32&   30&  28&   26 & 25 & 23 & 21&  20 & 18 & 16  &15 & 13 & 11 &  9 &  8&   6&   4&   3 &  1 & -0 & -2 & -3 & -5 & -7 \\
      18 &  33&   31&  29&   27 & 25 & 24 & 22&  20 & 18 & 16  &15 & 13 & 11 &  9 &  7&   6&   4&   2 &  0 & -1 & -3 & -4 & -6 & -8 \\
      19 &  34&   32&  30&   28 & 26 & 24 & 22&  20 & 18 & 16  &15 & 13 & 11 &  9 &  7&   5&   3&   1 & -0 & -2 & -4 & -5 & -7 & -9 \\
      20 &  35&   33&  31&   29 & 27 & 25 & 23&  21 & 19 & 17  &15 & 13 & 11 &  9 &  7&   5&   3&   1 & -1 & -3 & -5 & -7 & -9 &-11 \\ \hline
	\end{tabular}                                                                                                                     
\end{table}                                                                                                                         


% Initiative Roll Modifiers
% FILE Initiative Roll Modifiers
% REF tab:UnengagedInitiativeMods
\begin{stable}{Perception Modifiers}{ll}
\label{tab:PM}
	Situation				&  DF \\ 
\TableSubtitleRule
	Blinded				 &  -5 \\
	Deafened				 &  -3 \\
	Drunk/Stoned			 &  -5 \\
	Asleep					 &  -4 \\
	Poor Lighting			 &  -3 \\
	Not Alert 				&  -3 \\
	Alert					&  0  \\
	Actively Watching		& +3  \\ 
\end{stable}


The concept of suprise as such does not really exist as a seperate 
state. A character who is suprised is one who was not actively 
watching who got a poor roll at EF -3.

\section{Preset Reactions}

A character may decide to preset a reaction. A preset reaction is 
attempt to make yourself sensitive to a specific stimuli to the exclusion
of all other stimuli. The advantage is that it allows an additional EF +2 
to an initiative roll and a -3 to all subsequent initiative rolls. 
The disadvantage is that it adds EF -3 to any other perception roll. 
Gunfighters waiting on someone elses draw of a weapon would preset a
reaction. A character must be actively watching in order to preset a
reaction\footnote{This is necessary because you are dealing with a
reaction rather than an action.}.\footnote{When are presets declared?}

A Preset reaction may only be held for MST in the time scale 
that the players are working in before a drain roll is required.

If a character fails one of a series of preset actions he must roll 
EF-3 against GAW to avoid losing the series\footnote{Is a preset 
series broken or is it simply slower as a result of the need to 
recover? }.

\begin{quote}
Isaac (GAW = 20) expects to be attacked while walking in the alley. 
He states that he is actively watching for attack and , when he is 
attacked, he will dodge. When he is attacked he gets to roll 
initiative.  

Since he is actively watching he gets an EF +2 modifier to his 
initiative roll in addition to the EF +2 modifier for the preset. His 
total is \( EF = 7+2+2 = 11\). The Initiative Table shows a 4 in the 
EF=11, SB=10 position. He rolls a 5 so 5+4 = 9. So at count 9 Isaac 
will start to perform a dodge.
\end{quote}

\begin{quote}
Reginard (PAW = 18) is tucked around a corner in the alley listening 
for someone to mug. He is using only his hearing and is actively 
listening. When he hears someone just around the corner he is going 
to jump around the corner and slash at him with a dagger.

Because he is actively listening he gets EF +2. Because he is 
concentrating on PAW he gains an additional EF +2. Because he is 
using a preset he gets an EF +2. The total is \( EF = 7+2+2+2 = 13\).
The initiative table says that the Initiative modifier for SB=9 and
EF = 13 is 3. He rolls a 6 which, when added to the 3 modifier, gives 
a reaction of 9. So at 9 Reginard starts to do his attack.
\end{quote}


\section{Actions}

Actions normally begin at the count given by the initiative roll. The 
decision about what action is to be performed, if not already made,  
must be made at this point. The speed of the action is determined and 
the character takes this action on a pulse given by Initiative +
Action Speed.  

\begin{quote}
Rashid "the Twitch" (GAW=24) is walking past an alley when he notices the 
altercation in the alley. He is not actively watching (EF -3), so he 
rolls on the EF= 4 column. This gives him an initiative modifier of 
+10. Rashid rolls an 8 and is thus unable to start his action until 
count 18. 
\end{quote}

\section{Speeds of Actions}

Most actions have a speed associated with them. All simple actions 
, unless otherwise noted, have a standard speed of 5 counts. 
 
% Speeds of Basic Actions
% Speeds of Basic Actions
\begin{table}[hb]
\centering
\caption{Speeds of Basic Actions}
	\begin{tabular}{||l|l||} \hline
    Action						& Speed \\ \hline
	Lift Light object			&	5 \\
	Lift Heavy Object           &  10 \\
	Any Simple Physical Action  &   5 \\
	Perception					&	5 \\ \hline
	\end{tabular}
\end{table}


\begin{quote}
Isaac starts his dodge on 9, the speed of the dodge is 5 counts so 
Isaac starts his dodge on 9 and ends it on count 14. 
\end{quote}

\begin{quote}
Reginard  starts his jump and attack at 9, both actions take 5 counts 
and they are occuring simultaneously so the actions start at 9 and 
end on 14. 
\end{quote}

\section{Speeding up Actions}

Speeding up an action to 1/2 as long, causes the action to have 1/2 the 
Ef. Round to the Worst. At no point can an action take less than 1/3 
of its base speed.

\section{Drawing a Tool or Weapon}

This most often applies to drawing a weapon but can also apply to 
other tools.

In general, when a weapon is in hand, all normal weapon speeds apply. 

In order to get a weapon into ones hand it takes \( 2 \times 
Speed_{weapon} \) in pulses. 

In order to get a weapon in hand faster than \( 2 \times 
Speed_{weapon} \) requires a fast draw or ready roll against the weapon's 
skill. A successful ready roll brings the tool or weapon to bear at 
\( Speed_{weapon} \).  

\begin{quote}
When count 18 comes around Rashid attempts to fast draw his Isaac starts his dodge on 9, the speed of the dodge is 5 counts so 
Isaac starts his dodge on 9 and ends it on count 14. 
\end{quote}
 

\section{Resolving an Action}
Actions usually require only a skill roll to be made. 

\begin{quote}
The 
Isaac starts his dodge on 9, the speed of the dodge is 5 counts so 
Isaac starts his dodge on 9 and ends it on count 14. 
\end{quote}



\section{Resolving Multiple Actions}

For each ``Task'' there are a number of actions that can be performed. 
If the actions involved in the task are simple the GM may choose to resolve 
them with one roll rather than resolving each individual action. This may best 
be done when each action is simple ( i.e. EF >= 6). 

Jogging across the street and leaping a small fence is something suited to being 
a task. Normally though, the GM would not normally even ask the character to roll. 
a task roll simply because the actions are all very easy. But if the character 
stands the chance of being exposed to someone looking for him then a roll should
be made. 

The EF for the ``Task'' Roll 
is given by \[ EF_{Task} = 5 - ( 1/2 \times {Number\ of\ Actions }) \] 
The Gain Value for task roll is excellent success. This is a``Shifted Result''
roll. It is shifted to the less severe result. A Normal failure is a success,
a Normal Critical Failure is a Normal Failure, etc \dots

The experience points gained from a task roll match whatever the
final roll is. Those points may be distributed among any of the skills used
in the task.

\section{PFT Cost of Actions}

An individual can use a great deal of energy in performing actions in combat or
doing other simple tasks. For each period of activity \em{when the activity is over}
a drain roll is made. The drain roll is an EF=6 roll against PEN. The Loss is 
PFT based on what type of action he is engaging in.

All normal modifiers apply to this roll. So if someone is 50\% wounded in
PBD his Drain roll is made at \( EF = 7 - 3 = 4 \).  
\footnote{How to handle decay of Drain Roll EF with time}
\footnote{What are the time interval breakdowns for PFT loss} 
\footnote{Should there be drain rolls for each strike in combat ?}

% PFT Loss Numbers for a given activity
% FILE PFT and PEX cost for a given activity
% REF 
\begin{table}{PFT and PEX costs for activity}
	\begin{tabular}{lcc}

	Activity		&	PFT & PEX \\	
\hline
	Crawling		&	   &	  \\
	Walking			& 1/min	& 6/hr \\
	Jogging			&		& 1/min \\
	Running			&	   & 6/min \\
	Dash			&		& 2/sec \\
	Chopping Wood	 & 3/min & 18/hr \\
    \end{tabular}
    \caption{PFT and PEX costs for activity}
\end{table}


\section{MFT Cost of Actions}

An individual can use a great deal of energy in performing actions in studying or
doing other simple tasks. For each period of activity \em when the activity is over
a drain roll is made. The drain roll is an EF=5 roll against MEN. The Loss is 
PFT based on what type of action he is engaging in.

All normal modifiers apply to this roll. So if someone is 50\% wounded in
MBD his Drain roll is made at \( EF = 5 - 2 = 3 \).  

% MFT Loss Numbers for a given activity
% FILE MFT Loss Numbers for a given activity
% REF 
\begin{table}[h]
\centering
\caption{MFT Loss Numbers for a given activity}
	\begin{tabular}{lr} \hline
	Activity		&	LN	\\ \hline	
	Psionic Combat	&	8	\\
	Training		&	6	\\  \hline
	\end{tabular}
\end{table}



\section{Physical Movement}

Each character has a statistic named Physical Movement. This is the character's
movement in meters/second at a dash. There are a total of five different types
of movement that a character may utilize. Each type of movement has its own
movement rate which is derived from the character's movement statistic. 
Ideally the player will have the full range of movements listed on his 
character's sheet. 

% Include Movement Types Table
% FILE Movement Types
% REF 
\begin{SHTable}
	\begin{tabular}{ll}
	Movement Type		& Rate of Movement (meter/second) \\ 
\hline
	No Move             & \( 0 * Movement \) \\
	crawls, slow walks 	& \( 0.50 * Walk  \) \\
	Walking             & \( 0.50 * Jog   \) \\
	Jog					& \( 0.50 * Run   \) \\
	Run                 & \( 0.50 * Dash  \) \\
	Dash                & \( 1.00 * Movement \) \\ \hline
	\end{tabular}
    \caption{Movement Types}
\end{SHTable}


\begin{quotation}
As an example, Joe Daring has a PST of 16 and a PAG of 16. His movement is
\( (16 + 16) \over 4  =  8\). So Joe can Dash at 8 m/sec, run at 4 m/sec, jog
at 2 m/sec, walk at 1m/sec and crawl at .5 m/sec.
\end{quotation}

If the movement is being resolved during a time scale of greater than every
pulse one can get the distance traveled by simply multiplying the movement 
of the individual times the time spent moving. The time spent accelerating
is ignored as being negligible.

\begin{quotation}
Let us say that Joe Daring spends 15 seconds running down a deserted street.
If he does'nt run out of street he will have covered 4 * 15 = 60 meters. If 
this seems a bit short, keep in mind that a run is not a full dash. At a full
dash Joe would have covered twice the distance and would be slowing down pretty 
drastically due to losing wind.
\end{quotation}

\subsection{How to Handle Acceleration}

In dealing with movement on a pulse by pulse scale we need to actually deal
with acceleration. The sequence is quite simple. Whatever the final movement 
rate is that the character intends to use is considered the target movement rate.
When the character first starts moving he makes an acceleration roll in 
order to start moving at the movement rate just below the target movement rate.
Once the roll is made the character is now moving at that lower rate. On his next 
initiative the character may attempt to accelerate to the target movement. Note 
that the gain number is the movement rate. If an acceleration roll is failed
the end result is that the character drops to the next lowest available movement 
rate. Of course each of these acceleration rolls has its own EF modifier.  

% Acceleration Modifiers
% FILE Acceleration Modifiers
% REF 
\begin{table}[h]
\caption{Acceleration Modifiers}
\centering
	\begin{tabular}{ll} \hline
	Movement to Accelerate to	&	EF Modifier \\ \hline
	Dash						&	-5 			\\
	Run							&	-3			\\
	Jog							&	-2			\\
	Walk						&	0			\\
	Crawl						&	+2			\\ \hline
	\end{tabular}
\end{table}


\begin{quotation}
Reed Johnson has a movement of Dash 10, Run 5, Jog 2.5, Walk 1.3, Crawl .6
\end{quotation}

\subsection{Movement Modifiers}

% Targeted Action Movement Modifiers
% FILE Targeted Action Movement Modifiers
% REF 
\begin{SHTable}
	\begin{tabular}{ll}
	Slow move: crawls, slow walks (combat )   & DF -2 \\
	Normal move: Walking					   & DF -4 \\
	Double Move: jog						   & DF -6 \\
	Triple Move: Run						   & DF -8 \\
	Fast move: Dash						   & DF -10\\
	Vehicular Movement						   & DF -14 \\ 
	\end{tabular}
    \caption{Targeted Action Movement Modifiers}
\end{SHTable}


\section{Mental Movement}
This is a measure of the character's speed of mental travel. It is 
usually only used in Psionics and Computer usage.

\section{Opposing Skill Rolls}

An opposing skill roll in a roll in which the character attempts to 
undo an action done previously by another character. Typically the SN 
of the original action is taken as a negative modifier to the current 
skill roll.

\section{Stealth and Concealment}

Opposing Skill Rolls

\section{Deception and Detection}

Opposing Skill rolls

