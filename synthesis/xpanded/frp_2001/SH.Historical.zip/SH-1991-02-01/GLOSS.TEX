\chapter{Glossary}
A great deal of work has gone into making \SH\ both
easy to understand and to play. Unfortunately there is little
that a Science Fiction Role Playing Game can do to avoid
acronyms and abbreviations. This chapter describes the most basic 
and common of the acronyms, abbreviations and specialised 
terms that you will encounter over and over in the game rules. 

\begin{description}
	\item[PC] Player Character 
    The persona being acted out by the player
    \item[NPC] Non-player Character
    A persona being acted out by the Games-Master
    \item[GM] Games-Master
    The individual directing or guiding a game.
    \item[Statistic or ``stat'']
    The mathematiucal value representing a specific mental or 
	physical attribute of the PC or NPC.
	\item[Target Number (TN)] 
	The number that a character is trying to roll under. 
	Typically expressed as a percentile value.
	\item[Success Number (SN)] 
	The difference between the target number and the number actually rolled. In
	almost all cases the player wishes to roll below the target number so that:
	\(Success Number = target - roll\)
    \item[Critical Success]
	A roll with a Success Number greater than 50. This generally gives better results than 
	a normal success.
    \item[Critical Failure]
    A roll with a success Number less than -50. This generally gives poorer 
    results than a normally failed roll.
	\item[Open Ended Rolls]
	Rolls in \SH\ are open ended. If a high roll is made (00)
	then an additional roll is made and added to it. If a low roll is 
	made (01) then the next roll is added to the effect number of the 
	critical effect determined for this number. 
    \item[Shifted Results]\footnote{Should Shifted results exist ?}
	\item[Ease Factor (EF)] 
	The numeric Ease assigned a particular feat. The higher the Ease Factor, the
	easier the feat.
	\item[Difficulty Factor (DF)] 
	The numeric difficulty assigned a particular feat. Usually determined from
	the EF by \(DF = 10-EF\). The higher the DF the more difficult the feat. DF is
	always derived from EF and rarely directly used in calculations.
	\item[Stat Basis (SB)] \footnote{Clarify this definition}
	The combination of one or more stats required to use a skill or perform an
    action. The Stat Basis is half of the stat in question. If 
	multiple stats are mentioned the stat basis is 1/2 of the average 
	of the stats. Typically, when a stat basis is mentioned, it is referred 
	to by the name of the stat or stats that make up its final value.
	\item[Rank (RNK)] 
	The numeric level the character has in the skill.
	\item[Rank Bonus (RB)]
	The gain per rank in percentage chance to perform the skill
	\item[Skill Roll (SR)]
	A roll to perform a specific action or use a skill.
	\item[Learning Roll (LR)]
	A roll made to gain a rank in a skill.
	\item[Gain Number (GN)]
	The value that tells what the positive effect is when a skill roll or
	learning roll is made.
    \item[Loss Number (LN)] 
	The value that tells what the negative effect is when a skill roll 
	is failed.
	\item[Targeted Action] 
	An action in which the entity is trying to perform an activity 
	that requires concentration but no interaction between the target 
	and person performing the action.
\end{description}

\section{Formats}

In \SH\ the term ``Formats'' \footnote{Is there a better term ? }
has a very special second meaning. When discussing the manner in 
which a piece of important information is to be displayed the term 
format is used to represent that form of display\footnote{God, does 
this paragraph need help !}. 

When talking about the way in which missile weapons are listed in the 
\SH\ Instance Manual the rules will refer to the 
``Missile Weapon Format''. The Missile Weapon Format is simply the 
standard way in which any listing of missile weapons will be displayed.
All formats will be found in the \SH\ Paradigm.


