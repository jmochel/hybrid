\chapter{Rolling the Dice}

\section{Outline}
This chapter discusses the way in which dice are used in Space Hybrid.

\section{The One True Roll}
There is one main type\footnote{Excluding Rolls against a table}
of die roll in Space Hybrid. The roll is made
with percentile dice against a Target Number (TN). Normally the TN is
derived by multiplying the Ease Factor (EF) of the roll times the 
Stat Basis (SB) of the roll. The difference between the Target Number 
and the actual die roll is the Success Number (SN). To determine the 
effects of greater than average or less than average success numbers 
the (Optimistically named ) Success Table is used. If the Gain 
associated with the roll is numeric, the effect of the Success Table 
is multiplied by the Gain associated with the roll.  If the Gain 
associated with the roll has a non-numeric value, then the subjective 
result portion of the Success Table should be used and the GM has to 
make a judgement call. In SH, wherever we expect a subjective result 
to be typical, guidelines will be listed.\footnote{discuss loss 
numbers also} 

% Include the Success Table
% FILE Critical Table
% REF
\begin{stable}{Critical Success Table}{lll}
	Type of Success				& value & Subjective Value	\\
\TableSubtitleRule
	One Half \( 1/2 \)			& 1.25	& Solid Success		\\
        One Quarter \( 1/4 \) 	& 1.5	& Notable Success	\\
        One Tenth \( 1/10 \) 	& 2.0	& Very Notable Success	\\
        \(1/100\) 				& 3.0	& Amazing Success	\\
\end{stable}


\subsection{Examples}
As an example, let us look at a simple roll. The character must make a
ease factor 7 roll against a stat basis of 8. The target number is
thus: \(7*8 = 56\). The character rolls a 4 on percentile dice. The target
number - the roll is 52. So the success number is 52. This is enough to
bring us into the 1.5 effect of the success table. If there was a
specific numeric gain associated with this roll the character would
have achieved 1.5 times that gain. The numeric quantity gain is
referred to as Gain Number (GN).

Some typical rolls are displayed below.

\begin{itemize}
	\item An EF = 4 roll, SB = 7, TN = 28
	\item An EF = 10 roll, SB = 8, TN = 80
	\item An EF = 6 roll, SB = 5, TN = 30
\end{itemize}

\subsection{Calculating Succes}
In calculating the effects of a SN one simply takes the expected 
gain of the roll and multiply by the Effect Number on the right hand 
side of the table.

In the case of non-numeric gains it is up to the GM to decide 
what is the overall gain of the roll.




