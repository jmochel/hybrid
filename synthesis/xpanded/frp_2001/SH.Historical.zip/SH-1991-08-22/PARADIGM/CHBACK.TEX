\chapter{Background Generation}

\section{Outline}

This phase of character generation is where a majority of a character's
skills and history will be developed. 

In delineating a character's history it is assumed that he or she has
embarked on a course of career or non-career that brings him into
contact with chances to raise his skills, his status, and his material
wealth.

During the rolling of a characters career points gained are added 
and subtracted to or from four ``pools''. These four pools : Stats, 
Skills, Wealth, and Status are the basis of the final resolution of 
the character's skills, history, and station.

Each of the available paths has its own advantages and disadvantages.
Educational careers give one little chance to injure one's self but the
possible monetary gains are low.  Military careers are dangerous, but
possibly fairly rewarding.

The four pools each have a different basic function.

The Statistics pool or ``Stat'' pool, serves as the point pool for 
increasing a PC statistics or creating special abilities.

The Skill pool is the repository of points to be spent in gaining 
skills and skill packages.

The Wealth pool contains points to be spent in establishing the
character's basic financial state.

The Status pool contains points to be spent in gaining all the 
possible trappings of status: reputation, syncophants, or recognition.

Players do have a limited amount of lateral movement for these  points.

In addition a player can add and subtract to/from the pools by the 
usage of Advantages and Disadvantages.

\subsection{Careers}
Each career is delineated by a simple set of numbers combined with a
simple description. It has the following format:

\marginpar{
	{\em GMs Note} In real life careers are not balanced packages. 
But we must attempt to maintain game balance 
}

\begin{figure}[htb]
\caption{Career Format}
	\begin{tabular}{|l|} \hline
	Name:B.Skill:B.Financial:(Stat|Skill|Wlth|Status)\\  \hline
	Begin Detail \\
	\dots \\
	End Detail \\ \hline
    \end{tabular}
\end{figure}

\subsection{Career Format Explanation}
\begin{description}
	\item [Name]
	Self explanatory
	\item[Base Skill Points]
	The base skill points is how many skill points are normally
	recieved during a one year period of endeavor.
	\item[Base Financial Gain]
	The base amount of stads(standards) gained in a one year period of a career.
	\item[Career Ease Factors (Stat|Skill|Wealth|Status)]
	\item[Stat]
	The EF for a GAW roll to determine the overall effect the career
	had during a given time period on an entities health. Critical
	success usually allows the raising of a physical primary
	statistic. Critical failure effects usually involves the loss of
	points from a physical stat. The GN = 0.5. and is applied to the 
	Statistic Pool.
	\item[Skill]
	The EF for a GAW roll to determine whether or not the character
	gained any skills during this time period. The GN = Base Skill 
	Points associated with the career and is applied to the Skill 
	Pool. Critical failure effects do not apply to this	roll.
	\item[Wealth]
	The EF for a GAW roll to determine whether or not the character
	gained in the material or financial area. GN = Base Financial 
	Gain of the career. 
	\item[Status]
	The EF for a GAW roll to determine whether or not the character
	gained in the area of Status. Status is a rather subjective thing
	but typically, a military career leads to increases in rank and
	possible minor fame. A increase in status in a shadowy career would
	lead to the development of a "Rep". Obviously, the types of gain would
	need to be negotiated between the player and the GM. The GN is 1.
	\footnote{All careers should have a rank gain cost in status points}
	\item Typical Careers
	\begin{description}
		\item[Pick Pocket]
		4rnks:4,000stads:(5-6-4-6) 
		\item[Smuggler]
		5rnks:10,000stads:(4-7-5-3)
		\item[Terran Space Navy]
		6rnks:14,000stads:(4-7-3-3)
		\item[Grunt Mercenary]
		5rnks:7,000stads:(4-7-3-5)
		\item[Scouts]
		6rnks:4,000stads:(4-8-4-5)
		\item[Nurse]
		8rnks:12,000stads:(6-7-5-3)
		\item[Traffic Controller]
		6rnks:25,000stads:(5-6-3-2)
		\item[Advanced Education]
		8rnks:2000stads:(6-7-3-3)
	\end{description}
\end{description}

The example careers listed above would usually be fleshed out with 
additional detail such as a description of rank and status, etc\dots
\footnote{An important question is that of when a character is
allowed to drop out of military and so on careers.}
\marginpar{The GM will find that having the character keep track of 
the generation history is an aid in describing the character's 
history at a later time}
\marginpar{ Note that the Confederation Instance Manual has a full
description of many more careers}

\section{Designers Notes}

The nominal maximum number of points that may be gained in each one 
year resolution period is as below. This is not the maximum that the 
character may roll , but it is the maximum that may be a part of the 
design of a career.

\begin{itemize}
	\item[Wealth] 10 points
	Wealth values are table based as per career ? Should this be ?
	\item[Status] 2 points
	\item[Skills] 10 points
	\item[Health] 2 points
\end{itemize}

