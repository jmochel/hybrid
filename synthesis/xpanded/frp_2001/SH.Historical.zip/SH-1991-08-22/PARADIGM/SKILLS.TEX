\chapter{Skills}

\section{Outline}

This chapter discusses one of the most important parts of the system. 
Skills form the basis of almost every action a player character 
performs. It is discussed seperately from other subjects
because it is a broad topic and one that will need to be referred to 
many times.

\section{Definitions}

Proficiency in a skill is delineated by a range of ranks of 0-20. 

\section{Skill Format}

Skills have a simple format for describing them. The format described 
here is fairly detailed but most players will not need to concern 
themselves with most of the detail. 

Most players will only need to know the Name, the SB, the RB, and the cost 
of a skill.

\begin{figure}[h]
\centering
\caption{Skill Format}
	\begin{tabular}{|l|} \hline
	Name   \\
	Type = (Basic Type, Implementation, Interaction, Focus) \\
	Stat Basis \\
	Rank Bonus \\
	Cost \\
	Begin Detail \\
	\dots
	End Detail \\
	Typical Ease Factors \\ \hline
    \end{tabular}
\end{figure}

\section{Skill Format Explanation}
\begin{description}
	\item[Name]
	Self Explanatory
	\item[Type]
	The concept of type is detailed by a root word that describes the base
	type and modifiers that describes the level of other interaction. So
	typically a skill type would be Basic Type, Interaction, Implementation.
	A simple Ballroom Dancing skill would have a Skill Type of
	Art, Assisted, Non-Tool Based . See below for descriptions of the
	categories.
	\item[Stat Basis]
	The stat , or combination of stats, that represents the majority of the
	entities qualities utilized by that skill.
	\item[Rank Bonus]
	The additional gain in the success chance per rank in the skill. This can
	usually be inferred from the skill type but it is usually explicitly
	stated in order to save time and energy.
	\item[Cost]
	The experience point cost is the amount of experience points it takes to
	buy a roll in a skill. Typically , it also can be inferred from the type
	of the skill, but it is easier to have it explicitly listed.
	\item[Generation Cost]
	The cost in Skill Pool Points of a skill or skill package. Skills
	only cost 1 Skill Pool Point. Skill packages typically vary from 1
	to 10 Skill Points in cost.
	\item[Typical Ease Factors]
	This is just a list of things that characters might want to do with a
	given skill, and what the typical Ease Factor is for that occurence.
	\item[Basic Type]
	The overall term that described the very broad area of
	endeavour that the skill falls into. The basic type may be any of the 
	following:
	\begin{description}
		\item[Art ]
		An activity that has as its aim an affect upon the aesthetic senses of its
		audience.
		\item[Craft]
		An activity that is intended to mix Art with the production of some
		utilitarian object or effect.
		\item[Science]
		A series of disciplines intended to increase a codified body of
		knowledge.
		\item[Engineering]
		Any area of endeavour that attempts to apply a codified body of
		knowledge to the production of a desired physical effect.
		\item[Technical Study]
		An area of endeavour that is focused upon creating and maintaining the
		end result of the corresponding engineering discipline
		\item[Physical Discipline]
		Any area of endeavour based on muscle memory training.
		\item[Mental Discipline]
		Any area of endeavour based on purely mental manipulations without 
		reference ...\footnote{How in the world do I describe this one}
		\end{description}
		\item[Interaction]
		This area describes the various modifications upon the broad themes
		described above. Interaction may be any of the following:
		\begin{description}
			\item[Assisted]
			Requires the work of one or more additional entity.
			\item[Un-Assisted]
			Requires only one entity.
		\end{description}
	\item[Implemetation]
	Implementation describes any tools that need to be utilized in order 
	to perform an action using the skill. It is based on the most common
	usage of a skill. It can be any of the following.
	\begin{description}
		\item[Complex Tool Based]
		Requires some additional tools of a moderately complex nature that have been
		made especially for the skill.
		\item[Simple Tool Based]
		Requires some additional tools other than the entities mind or body. 
		Computers and calculators are good examples.
		\item[Non Tool Based]
		Requires no additional tools other than the entities mind or body.
	\end{description}
	\item[Focus]
	Focus is only associated with the sciences and it refers to whether 
	the character studies the full science or specializes in some sub 
	section of that science.
	\begin{description}
		\item[Non Directed]
		Intended to add to a general body of knowledge
		\item[Directed]
		Intended to add to a specific area of a body of knowledge.
	\end{description}
\end{description}

\section{Using Skills}
\subsection{Making A Skill Roll}

To make a skill roll involves determing the Ease Factor for the action
you wish to perform, multiplying it by the Stat Basis of the applicable
skill, and adding in the Rank Bonus * Rank. In other words, Acrobatics
is based on on PAG, for a SB of PAG/2. For an average individual that
is a 7. Assuming that the  character is attempting a simple forward
roll the EF is 7. \(7*7 = 49\%\). If the character is RNK 4 in the
skill then  the overall chance to perform the action is \(49 + 4 * 5 =
69\%\). All skills have a rank bonus of 5%

\subsection{Modifiers to Using Skills}
Obviously, a great deal of the work involved in determining the
percentage chance to perform a skill is dependent on the EF assigned to the
action. Most modifiers to that action will be added or subtracted from
the actions EF. There is a wide range of modifiers that can affect the EF 
of the skill roll\footnote{Add rules about the players being told the 
EF. The concept of faint unease enters here}.

\subsubsection{Knowledge of the skill}
A character may be attempting to use a skill he has no familiarity 
with such as firing a pistol for the very first time. At such moments 
the character is subject to modifiers to his action for his lack of 
knowledge regarding that skill. Having no rank in a skill gives EF -3
on the skill roll\footnote{Note that this is different from not 
having familiarity with the specific tool you are dealing with}.

\subsubsection{Setup}
The term setup means to wait and prepare for a given action. Typically it
takes as long to set up for the action as it does to perform the action itself.
This normally adds double the normal rank bonus to the percentage chance
to perform the skill as well as adding 1 to the EF of the action.

% Include General Modifiers Table
\begin{table}[hb]
\caption{General Modifiers}
\centering
	\begin{tabular}{||l||l|l||} \hline
	Situation			& EF Modifier	& Other Modifier \\ \hline
	No rank in skill 	& -4			& 				 \\
	Setup				& +2			& 2*RB			 \\ \hline
	\end{tabular}
\end{table}


\subsubsection{Physical Condition}
The character may attempt to use skills when he is injured or 
fatigued. Physical fatigue has the greates effect on physical actions 
but can also effect mental activities.

% Include Physical Condition Modifiers Table
\begin{table}[hb]
\caption{Physical Condition Modifiers}
\centering
	\begin{tabular}{||l|l||} \hline
	Situation			& EF Modifier	\\ \hline
	Out of PFT			& -2 physical	\\
	Out of PFT			& -1 mental		\\
	25\% wounded in PBD & -2 Physical	\\
	25\% Wounded in PBD & -1 Mental		\\
	50\% Wounded in PBD & -4 Physical	\\
	50\% Wounded in PBD & -2 Mental		\\ \hline
	\end{tabular}
\end{table}


\subsubsection{Mental Condition}
Much as physical condition affects the physical actions, so too does 
mental condition affect mental actions. Mental condition can also 
affect the efficiency of physical actions.

% Include Mental Condition Modifiers Table
% FILE Mental Condition Modifiers Table
% REF 

\begin{stable}{Mental Condition Modifiers}{ll}
	Situation			& DF Modifier	\\
\TableSubtitleRule
	Out of MEX 			& -6 mental 	\\
	Out of MEX			& -3 physical	\\
	25\% wounded in MBD & -2 Mental		\\
	25\% wounded in MBD	& -1 Physical 	\\
	50\% wounded in MBD & -4 Mental		\\
	50\% wounded in MBD & -2 Physical	\\
\end{stable}


\subsubsection{Movement} When performing a physical action the
character may be affected by  his rate of movement or the rate of
movemnt of some other object. If the character is using physical
movement greater than a walk the modifier due to movement is 
culmulative with normal florentine modifiers. 

\subsubsection{Environmental Conditions}
This is a catchall area. Characters generally are at their best 
performance in conditions similiar to the environment in which they 
were raised. Any drastic modifications from that environement in 
terms of light, gravity, humidity, etc\dots can lower the character's 
performance.

% Environmental Condition Modifiers
% FILE Environmental Condition Modifiers
% REF 
\begin{table}[h]
	\begin{tabular}{ll}

	Situation			& DF Modifier	\\
	\hline
	Lighting 50\% off	& -3			\\
	Lighting 75\% off	& -4			\\
	Gravity 50\% off	& -3			\\		   
	Gravity 100\% off	& -4			\\

    \end{tabular}
	\caption{Environmental Condition Modifiers}
\end{table}



\subsection{Relations Among Skills}

In situations where the character does not possess a skill that
directly relates to the action to be performed the entity may choose to use
a related skill.

A typical example would be in using two different types of handguns. The
character has rank 10 in Slug Pistol but is using a Stun Weapon. The stun
weapon is fairly different from the Slug Pistol so the character can only
apply 1/5 of his expertise in Slug Pistol to using this pistol. So he has
an effective rank 2 in the weapon.

As a rule the following relations apply.

% Skill Relations Table
% FILE Skill Relations Table
% REF 
\begin{stable}{Skill Relations}{ll}
		Similiar in many respects     &       2/5 \\
		Dissimiliar in many respects  &       1/5 \\
		Really Stretching it		  &       1/10 \\ \hline
\end{stable}


\subsection{Unfamiliar Tools}

If the skill requires the use of tools and the tool that the character is
utilizing is unfamiliar, then the action occurs at a -2 EF. This usually
only happens if the differences between the version of the tool the
character normally uses and the current one actual effect how it is
used. A gun with a different mass than the entity is used to is
unfamiliar, whereas a gun of the same model and same manufacturer is
not. To eliminate this unfamiliarity modifier requires that the entity
famaliarize himself with the tool with a EF -3 roll against the SB of 
the skill with a gain of +0.5 EF. The character may never reduce the 
unfamiliarity modifier below -0.5 unless he takes a skill that 
requires that tool.

\section{Gaining Skills}

There are two ways to gain skills. The first is to use a skill and have
its usage give you enough experience points to buy a roll for that
skill. The Second is to use general experience points to use in
training for that skill.

\section{Advancing in Skills}

Advancement is always due to EPS. Character gains EPS. Spends EPS in order
to get a roll in his skills. The cost of going up in a skill is listed with
the skill.         

\subsection{Experience Points Gain from usage.}

\begin{relate}
	\item[Non-Ranked] Critical gains 15 experience. Normal Usage gains 10 exp.
	\item[Ranked] Critical gains 25. Normal usage gains 15.
\end{relate}

\subsection{Learning Rolls}

A learning roll is made when the character has accumulated enough
experience to raise in a skill. At the point the character spends 
his experience points they may roll to determine if their rank in that
skill goes up.
 
The EF for making a Learning Roll in a skill is 

\((10 - Rank) + ({Training Mods})\)

This is multiplied by the SB of the skill in order to get the
percentage chance to go up in the skill. All the normal success table 
results apply.

Each time the character fails they gain an EF +1 to the next learning
roll they make in that skill. The modifiers are culmulative. Thus
someone who has failed the roll four times has an EF +4 modifier.

\subsection{Training in Skills}

Training in a skill directly modifies the EF of the Learning roll in
that skill. Training Alone adds \(+.1 EF/10 hours\). Training under a
Teacher adds \(+(.1 \times {Teachers Rank})/10 hours\). Training with
notes or study aids adds \(+(.05 \times {Teachers Rank})/10 hours\).

\section{Designing Skills}

If a character wishes to learn a skill not supplied by the rule 
system the GM must figure out the appropriate values for that skill. 
Typically the GM will break the skill down into its basic type, 
implementation, interaction, and focus. After doing so he will use 
that breakdown to calculate its experience point cost.

The first step is to determine the skill's stat basis. In order to do 
so the GM must decide what action comprises 50\% or more of the activity
the character undertakes when using that skill. Once he has determined 
what that action is he must decide what stats are used by that activity.

In general each skill is measured against the tertiary stats first , then the 
secondary and finally the primary. MSE and PSE are never used as SB's.
MVM based skills are self evident. 

Most scientific or purely mental skills are FCS based.
Lovemaking is PCA.

For actions in which the character is non skilled the SB's are fairly
simple. For perception rolls use GAW or MAW or PAW. For analysis rolls
use MCA for entity manipulated situations. Manipulated means that there
was a deliberate attempt on the part of someone else to direct someones
perceptions of this area. Use FCS for non-entity manipulated
situations. For memory rolls roll MST. 

If the most common action is a mental one but has a physical component then
the skill is not based on FCS,MCA, or MDF alone.

If the most common action is a physical skill but has a mental component then the
skill is not based on ACC,PCA, or PDF alone.

Skills that are not based on the tertiary stats are usually either MAW
or PAW. Any skills where the most common action is a perception roll has
to be based on one of the awareness stats. If the action could not be
performed at all if the entities primary physical sense were removed
then the skill is PAW based\footnote{Is the last really true ?}. 

% Include Generic EFs for Actions
% Generic EFs for Actions
\begin{table}[hb]
\centering
\caption{Generic EFs for Actions}
	\begin{tabular}{||l|r||}	\hline
	Basic Identification of Actions Needed		  & 10	\\ \hline
	Judgement of Quality                          & 9	\\ \hline
	Basic Perception Roll w/in area of SKill      & 8	\\ \hline
	Basic Action ( makes up 60\% or more of the   & 7	\\  
	actions made by someone using this skill)     &      \\
	Anyone of basic compentence would know this   &      \\
	action well.                                  &      \\ \hline 
	\end{tabular}
\end{table}


\subsection{Experience point costs}

% Include Costs of Skill Components Tables
% FILE Costs of Skill Components
% REF 
\begin{SHTable}
	\begin{tabular}{lll}
    BasicType           & Cost          & DF \\
	\hline
	Mental Disciplines      &       3         & -4 \\
	Art                     &       3         & -2 \\
	Science                 &       7         & 0 \\
	Engineering             &       5         & 0 \\
	Crafts                  &       4         & -1 \\
	Technical Study         &       4         & 0 \\
	Physical Disciplines    &       2         & 0 \\
    \hline
    Interaction                    & Cost          & DF \\
	\hline
	Unassisted              &       0         & 0 \\
	Single Assisted         &       1         & -1 \\
	Multiple Assisted       &       2         & -2 \\    \hline
    Tool Use                       & Cost          & DF \\
	\hline
	Non-Tool Based          &       0         & 0 \\
	Simple Tool Based      &        1         & -1 \\
	Complex Tool Based      &       2         & -2 \\    \hline
	\end{tabular}
    \caption{Costs of Skill Components}\label{Table:SkillComponentCosts}
\end{SHTable}




Note that the Appropriate Tech Index that the skill was learned at 
must be recorded.\footnote{ How to handle upgrading skills for new TI's ?}

\section{Filter Skills}

There is a category of skills which affects the use of other skills 
in an environment they were not designed to be used in. These skills 
are called filter skills. A Filter skill is any skill that can allow 
for the full expression of other skills in an environment other than 
that for which those skills were designed for.

Typical filter skills include the following: 0-g maneuver, Tech 
Level Lore, Culture Lore, Horsemanship and other vehicular combat 
skills, Armor Wearing, and Computer operations.

For situations in which the character is attempting to apply a skill 
in a environment he is not familiar with and that skill {\em must }
interact with that environment, then the rank in the filter skill 
becomes the upper limit on the effective rank of the skill being 
used.

As an example, if someone has a horsemanship skill at rank 5, he or she
may use their archery skill up to rank 5 without making any rolls
against their horsemanship. If the character has a higher archery skill
and wants to bring it all to bear on a shot, they must roll against
their horsemanship first in order to get the full use of the archery
skill.
 
\chapter{Skill Packages}

\section{Introduction}

What we have discussed up till now has been single skills.
Quite often though it may be better to offer skill packages.
A skill package is a collection of conceptually related 
skills that have an experience point cost less than the total 
cost of all the skills. All skills in a skill package may be 
used normally with the exception that package skills have a lower
Rank Bonus and EF modifier.

\section{Experience Point Cost}

Each skill package has an overall Experience point cost that is
the result of adding the cost of the most expensive skill together
with the cost of all the other skills times some factor. The actual value
of this factor is used to name the type of skill package.


{\bf Quarter Experience}
\[ 1 \times Most\ Expensive\ Skill + 1/4 \times All\ Others \]
All skills within this package have a RB = 2\%, EF = -2 

{\bf Half Experience }
\[ 1 \times Most\ Expensive\ Skill + 1/2 \times All\ Others \]
All skills within this package have a RB = 3\%, EF = -2

{\bf Three Quarter Experience}
\[ 1 \times Most\ Expensive\ Skill + 3/4 \times All\ Others \] 
All skills within this package have a RB = 4\%, EF = -2

\section{Making A Skill Roll} 

Skill rolls for skills within a package are made as normal with the
appropriate modifiers for the package skills.

\section{Modifiers to Using Skills}
All normal modifiers that apply to single skills apply to skills in a package.

\section{Learning Rolls}
The SB of the learning roll for a skill package is the least favorable 
of all the SBs in the skills that make up the package. A skill package 
suffers an additional EF -1 to Learning Rolls.

\section{Modal Packages}

Modal packages ( pronounced M\=o-dal ) are packages of Mental or Physical disciplines 
that require concentration to maintain, leading to the exclusion of any action that
requires skills outside the package. Many of the martial arts styles are modal
packages.  Modal packages do not have the EF = -2 modifier that most packages do
but they require a skill roll to enter and exit the mode of concentration.

\section{Unfamiliar Tools}
All normal modifiers for unfamiliar tools apply to package skills.

\section{Gaining Skills}
Skill Packages may only be gained by training in the package from
someone who already has that skill package. \footnote{ Include rules for 
actually designing skill packages. }

\section{Experience Points Gain from usage.}

All normal experience point gains from usage apply to skills in a skill
package with the exception that a character may never get eeps for
being non-ranked in a skill package\footnote{Put in notes on buying 
out skills from a package}. 

A typical skill package would be :

\begin{verbatim}
Aikido (3/4 package)

	Dodging In       (SB=PAG)
	Dodging Away     (SB=PAG)
    Grapple	         (SB=PCA)
    Balance Throw    (SB=PCA)
    Joint Throw      (SB=PCA)

	Cost ( 50 + 3/4(50+50+50+50) ) = 200
    Generation Cost = 1+3/4(1+1+1+1) = 4

	Skill PAckage SB: PAG|PCA
	RB=4%

\end{verbatim}

