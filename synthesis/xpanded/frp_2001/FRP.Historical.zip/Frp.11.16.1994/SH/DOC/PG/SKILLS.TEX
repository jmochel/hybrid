\chapter{Tasks and Skills}

A task is an action or a group of actions. To do a task the 
character determines the difficulty {\em Difficulty Factor} of the 
task and what skill(s) may be used to do the task. A base chance to 
{\em Base Chance} is determined and modified by the difficulty factor 
of the task. 

Forcing a locked door is a task that has some difficulty. If the 
character has no skill in forcing doors then they are forcing the 
door based on using just physical strength. Their chance to force the 
door is based on their physical strength and how difficult the door 
is to force. The sum total chance to force the door is called the 
{\em Success Chance}.

If the character has a skill in forcing doors then they will have 
knowledge about how best to apply their physical strength to get the 
door open {\em Skills}. 

\section{Tasks}

\subsection{Description}

\begin{description}
	\item[Name] 
	Self Explanatory
	\item[DF] The difficulty of the task
	\item[SB] The stat basis of the task
	\item[Time]
    How long the task typically takes
    \item[Applicable Skills]
    Any skills that may be applied to the task
\end{description}

\subsection{Difficulty Factors}

The difficulty of a task is described by a number 
referred to as an ``Difficulty Factor'' or DF. Difficulty Factors 
for tasks typically range from -10  to +4. Throughout \SH\ it is 
assumed that the base DF of an action is 0 {\bf unless otherwise 
stated}. 

The modifier for a task is simply 5\% times the Difficulty Factor or:
\[ Modifier = 5 \times {Difficulty\ Factor} \]

If there are a series of simple actions (DF 0) that can be lumped 
together in a single task the DF for the task is given by \[ DF_{Task} =  - ( 1
\times {Number\ of\ Actions }) \]  

Jogging across the street and leaping a small fence are actions that 
are best lumped together into one task. There is no reason 
to ask the character to roll a task roll for each action. 
But if the character stands the chance of being exposed to
someone looking for him then a roll should be made for the entire set 
of actions. 

\begin{table}[h]
	\begin{tabular}{l|l}
	Subjective						& DF \\
	\hline
	Trivial			&  0 \\
	Non-Trivial	 	&  -1 \\
	Difficult		&  -3 \\
	Very Difficult  & -5 \\ 
	Damned Difficult & -7 \\
   	Nearly Impossible & -10 \\
	\end{tabular}
    \caption{Generic Difficulty Factors}
\end{table}

\subsection{Stat Basis}

The task has a stat basis that describes what stat or combination of 
stats can be used to do the task. This is only used if the character 
has none of the skills in the Applicable Skills entry. 

The Base Chance for someone who has no skill is 
\[ (3 \times SB_{skill}) \over {2} \]. 

\subsection{Time}

The task will have time associated with it. This is the average time 
the task typically takes to perform. 

\subsection{Applicable Skills}

This is a list of suggested skills that could be used to do the task. 
It is not exhaustive.

\section{Skills}
\subsection{Description}

\begin{description}
	\item[Name] 
	Self Explanatory
	\item[Stat Basis] 
	The stat, or combination of stats, that is used by the skill. 
	\item[Difficulty Factor] 
	The modifier for doing any action with this skill.
	\item[Generation Cost]
	The character generation cost of a skill or skill package. Skills
	only cost 1 point. Skill packages typically vary from 1
	to 10 points in cost.
	\item[EP Cost] 
	The experience point cost is the amount of experience points it takes to
	buy a roll in a skill. 
\end{description}

\subsection{Ranking}

Proficiency in a skill is described by a number with a range of 0-20.
The higher the number, the greater the character's expertise. Someone
is completely unfamiliar with a skill is considered to be
\ndx{unranked}. Someone who is familiar with the basics of
the skill is rank 0. A dedicated amateur is has a rank 4-6. A
solid workaday craftsman has a rank 7-9. A dedicated professional is rank 
10-13. An expert in a skill has a rank 14-16. A true master of a skill has 
a rank 14-20 in a skill. 

\subsection{Stat Basis}

Each skill has a stat or a combination of stats that is called the 
stat basis and is used to calculate the base chance of using the 
skill. 

To use a skill the GM determines what the Base Chance of the skill is 
and adds in the modifiers for the task being performed.
The Base Chance of using a skill is three times the Stat Basis of the skill
or \[ 3 \times SB_{skill} \] For each rank the character has in the skill add 4\%.
The modifiers for the task vary for each situation.

\begin{quote}
A character with rank:0 in Rock Throwing is throwing a rock 
across the street. The character has a Accuracy (ACC) of 12. Their Base Chance 
to hit is \( 2 \times 12 = 24\% \).
\end{quote}

\subsection{Generation Cost}

When generating the character the player spends points from a
personal development pool to buy skills. The generation cost is the 
cost in personal development points for the skill. The first rank of 
a skill (rank 0) costs twice the generation cost and each rank 
thereafter costs the generation cost. 

\subsection{Experience Point Cost}

Once a character has been generated skills are gained or increased by 
spending experience points. The first rank of a skill (rank 0) costs 
twice the EP cost and each rank thereafter costs the EP cost. 

\subsection{Gaining Skills}

A character can gain experience points for roleplaying and use those 
experience points to buy a learning roll for that skill. The 
experience point cost to buy a learning roll is listed with the 
skill. Once the character spends those experience points they roll to 
see if their rank in that skill goes up.

The Success Chance for a learning roll is the Success Chance of the skill with 
the DF being: 

\[DF_{Learning\ Roll} = (-Rank) + ({Training\ Mods}) \]

to put it into a task description:

\begin{verbatim}
	Task: Increase the skill level
	DF: \[ (-{Skill\ Rank}) + ({Training\ Mods}) \]
	SB: The stat basis of the skill
	Time: Instantaneous
    Applicable Skills: None
	Notes: DF +2 for each previously failed roll on this skill.
\end{verbatim}

If the character fails to make the learning roll they gain a Df +2 to the next 
learning roll in that skill. The failure modifier is culmulative. A character
that has failed four times will get a DF +8 modifier to their next earning roll
in that skill.

\subsection{Training in Skills}

Training in a skill directly modifies the EF of the Learning roll in
that skill. Training Alone adds \[+1 DF/20 hours\] 
Training under a Teacher adds \[+(1 \times {\Delta Teachers\ Rank})/20 hours\]
Training with notes or study aids adds \[+(1 \times {Teachers 
Rank}/2) \over {20\ hours}\]

\subsection{Relations Among Skills}

In situations where the character does not have a skill that
is directly applicable to the task being performed the character 
may choose to use a related skill.

A typical example would be in using two different types of handguns. The
character has rank 10 in Slug Pistol but is using a Stun Weapon. The stun
weapon is fairly different from the Slug Pistol so the character can only
apply 1/5 of his expertise in Slug Pistol to using this pistol. So he has
an effective rank 2 in the weapon.

As a rule the following relations apply.

% FILE Skill Relations Table
% REF 
\begin{stable}{Skill Relations}{ll}
		Similiar in many respects     &       2/5 \\
		Dissimiliar in many respects  &       1/5 \\
		Really Stretching it		  &       1/10 \\ \hline
\end{stable}


\subsection{Unfamiliar Tools}

If the skill requires the use of tools and the tool that the character is
utilizing is unfamiliar, then the action occurs at a DF -2. This usually
only happens if the differences between the version of the tool the
character normally uses and the current one actual effect how it is
used. A gun with a different mass than the entity is used to is
unfamiliar, whereas a gun of the same model and same manufacturer is
not. To eliminate this unfamiliarity modifier requires that the entity
familiarize himself with the tool with a DF -3 roll against the SB of 
the skill with a gain of 1 DF per roll. 

\section{Types of Skills}

The section on skills describes the basic way that skills are 
handled but there are a variety of special types of skills that are 
used for special circumstances.

\subsection{Specialization}

Skills that are described as general skills cover a wide range of 
tasks with very little depth. A person who has learned a general 
skill such as Throw Object is able to throw just about anything they 
can get their hands on ( knives, spoons, rocks, chairs) with a lesser 
success chance than someone who has a specific skill in throwing a
particular object.

In addition, there are skills known as support skills that are solely 
designed to increase the success chance when doing one type of action 
with a skill. Someone who uses their sword to parry weapon attacks 
may wish to train specifically in parrying with a sword. So they 
would have a ``Long Sword' skill and a ``Long Sword : Parry'' skill.


General skills only give 1\%/rank to the success chance. Specific 
skills (the \SH\ norm), give 4\%/rank. Support Skills add 2\%/rank.

\subsection{Filter Skills}

There is a category of skills which affects the use of other skills 
in an environment they were not designed to be used in. These skills 
are called filter skills. A Filter skill is any skill that can allow 
for the full expression of other skills in an environment other than 
that for which those skills were designed for.

Typical filter skills include the following: 0-g maneuver, Tech 
Level Lore, Culture Lore, Mounted Combat and other vehicular combat 
skills, Armor Wearing, and Computer operations.

For situations in which the character is attempting to apply a skill 
in a environment he is not familiar with and that skill {\em must }
interact with that environment, then the rank in the filter skill 
becomes the upper limit on the effective rank of the skill being 
used.

As an example, if someone has a mounted combat skill at rank 5, he or she
may use their archery skill up to rank 5 without making any rolls
against their mounted combat. If the character has a higher archery skill
and wants to bring it all to bear on a shot, they must roll against
their mounted combat first in order to get the full use of the archery
skill.

