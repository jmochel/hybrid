\chapter{Background Generation}

\section{Outline}

This phase of character generation is where a majority of a
character's skills and history will be developed..

\section{History}

In delineating a character's history it is assumed that he has embarked
on a course of career or non-career that brings him into contact with
chances to raise his skills, his status, and his material wealth.

Each of the available paths has its own advantages and disadvantages.
Educational careers give one little chance to injure one's self but the
possible monetary gains are low.  Military careers are dangerous, but
possibly fairly rewarding.

\subsection{Careers}
Each career is delineated by a simple set of numbers combined with a
simple description. It has the following format:

\begin{figure}[htb]
\caption{Career Format}
	\begin{tabular}{|r|} \hline
	Name:Res Period:B.Skill:B.Financial:(Stat|Skill|Wlth|Status)\\  \hline
	Begin Detail \\
	\dots \\
	End Detail \\ \hline
    \end{tabular}
\end{figure}

\subsection{Career Format Explanation}
\begin{description}
	\item [Name]
	Self explanatory
	\item [Resolution Period]
	The time between resolution rolls. Typically one to two years.
	\item[Base Skill Points]
	The base skill points is how many skill points are
	normally recieved for that time period of endeavor.
	\item[Base Financial Gain]
	The base amount of stads(standards) gained for a given Resolution
	period of a career.
	\item[Career Ease Factors (Stat|Skill|Wealth|Status)]
	\item[Stat]
	The EF for a GAW roll to determine the overall effect the career
	had during a given time period on an entities health. Critical
	success usually allows the raising of a physical primary
	statistic. Critical failure effects usually involves the loss of
	points from a physical stat. The GN = 0.5. and is applied to the 
	Statistic Pool.
	\item[Skill]
	The EF for a GAW roll to determine whether or not the character
	gained any skills during this time period. The GN = Base Skill 
	Points associated with the career and is applied to the Skill 
	Pool. Critical failure effects do not apply to this
	roll\footnote{Should this be a shifted result}

	\item[Wealth]
	The EF for a GAW roll to determine whether or not the character
	gained in the material or financial area. GN = Base Financial 
	Gain of the career. 

	\item[Status]
	The EF for a GAW roll to determine whether or not the character
	gained in the area of Status. Status is a rather subjective thing
	but typically, a military career leads to increases in rank and
	possible minor fame. A increase in status in a shadowy career would
	lead to the development of a "Rep". Obviously, the types of gain would
	need to be negotiated between the player and the GM. 
	\footnote{All careers should have a rank gain cost in status 
	points}
	\item Typical Careers
	\begin{description}
		\item[Pick Pocket]
		1yr:4rnks:4,000stads:(5-6-4-6) 
		\item[Smuggler]
		1yr:5rnks:10,000stads:(4-7-5-3)
		\item[Terran Space Navy]
		3yr:6rnks:14,000stads:(4-7-3-3)
		\item[Grunt Mercenary]
		1yr:5rnks:7,000stads:(4-7-3-5)
		\item[Scouts]
		1yr:6rnks:4,000stads:(4-8-4-5)
		\item[Nurse]
		2yr:8rnks:12,000stads:(6-7-5-3)
		\item[Traffic Controller]
		1yr:6rnks:25,000stads:(5-6-3-2)
		\item[Advanced Education]
		1yr:8rnks:2000stads:(6-7-3-3)
	\end{description}
\end{description}

The example careers listed above would usually be fleshed out with 
additional detail such as a description of rank and status, etc..
\footnote{An important question is that of when a character is
allowed to drop out of military and so on careers.}
\footnote{PCs should keep track of the generation history as an aid 
in designing the character's history}

\section{Designers Notes}

The nominal maximum number of points that may be gained in each one 
year resolution period is:

\begin{itemize}
	\item[Wealth] 10 points
    Wealth values are table based as per career ? Should this be ?
	\item[Status] 2 points
	\item[Skills] 10 points
	\item[Health] 2 points
\end{itemize}
