\documentclass{book}

\usepackage{makeidx,multicol}
\usepackage{sh}

% General Parms

\hbadness=10000					% do not want underfull box messages
\hfuzz=\maxdimen				% no overfull box messages either

\topmargin=0in
\textheight=9in					
\textwidth=4.8in				
\headsep=20pt
\marginparsep=.2in
\marginparwidth=2in

\raggedbottom

% Title Page

\title{Fantasy/Space Hybrid \thanks{Copyright 1997}: PSI Docs}
\author{Jim Jackl-Mochel}
\date{\today}

\begin{document}


% The TOC

\pagenumbering{roman}
\tableofcontents
\listoftables

% Back to normal

\pagenumbering{arabic}

% The introduction

\chapter{Psionics}

\section{Intro}


This chapter outlines one of the more interesting areas of SF gaming: 
Pyschic Phenomenom or Psionics. What is to be noted is that no 
comfortable way could be found to make a truly generic set of psionic 
abilities. A great deal of the limitations described here are indeed 
implementation specific and could be changed on GM's whim.

Psionics is the application of mental force to directly affect an entitie's
surroundings. The stat PSI measures how much power the psionic can access and control
and skill in disciplines determines how well it can be molded to their will.

\section{Areas of Psionics}

Psionics are broken down into categories by the general type of
manipulation that is performed.

\begin{description}

    \item[Telepathics]

    The use of the mind to influence other minds at the emotional and concious
    communication level. This does not go down far enough to effect most bodily
    processess. Telepathics are generally of low cost, long range, and
    easy difficulty. Empathy, Surface Read, and Telepathy are typical
    examples of Telepathic skills.

	\item[Telekinetics]

	Telekinetics is the use of the mind to impart kinetic energy to an
	object. The use of psionics to toss objects around is a telekinetic 
	action. Telekinetic skills are typically of medium cost, medium
    difficulty, and short Range. Telekinesis is a Telekinetic skill.


	\item[Temporokinetics]

	Temporokinetics is the use of the mind to affect the spatial integrity
	of the surrounding area. Temporokinetic skills are of very high cost,
    high difficulty, very short duration. Teleportation is a temporokinetic
    action.

	\item[Biokinetics]

	The use of a mind to directly influence a biological system.
    Biokinetics actions are typically of Med Cost, Medium Diff, and
    Short Range. Placing oneself in suspended animation is a
    typical biokinetic feat.

	\item[Energetics]

	The use of the mind to directly affect the molecular nature of a
	material. Energetic actions are typically of  High Cost, Medium Diff,
    Short Range. Setting fire to paper or dissolving a glue would be 
	typical actions of an energetic.

	\item[Mystokinetics]

    Basically this covers the more mysterious skills that are less subject
    to mechanistic explanations than the above four categories. 
    Typically the Mystokinetic actions are of Low Cost, Very High Difficulty,
    and medium duration.
    Precognition, Clairsentience, and Object Read are typical
    Mystokinetic skills.

\end{description}

\section{Gaining Psionics}

All entities have the potential for psionics. Most will not have the
training to use or access psionics. Characters that take the Active
Psionics enhancement (often part of a racial package ) will have
recieved concious training or have been raised in a high-psi environment
and absorbed an instinctual level of usage.

All Psionic Skills are mental disciplines.

\section{Usage of Psionics}

There are several things of note to be mentioned. The first and foremost
is that it is always easier to affect inanimate objects with Psionics.

When attempting to affect other entities the percentage chance to succeed
is lowered by the defending entities MDF. An actively defending individual
lowers it by 2 * MDF. In addition, there are a number of situations
(critical misses, instinctively defending opponents) that could cause
the psionic great pain when attempting to affect a living creature.

\subsection{Skill Rolls}

Attempting to use Psionics requires a skill roll just as any other skill.
The same DF modifiers apply as to any other action. 

\subsection{Concentration Aids/Checks}

If the Psionic is disturbed in any way the psionic must make an DF -2 roll
against that psionic skill in order to continue their concentration. This
concentration may be made up to three times per pulse. If more than that is
needed , give it up. Psionics may use special aids in concentrating including: 
Meditation, Ear Plugs, Drugs etc...

\subsubsection{Meditation}

Meditation allows the psionic to apply \[ 2\%/rank\] for their Meditation skill to their chance
to perform the discipline. It cuts all non-psi perceptions to  \[ 1/2\]
their normal chance.

\subsubsection{Foci}

The use of Foci may be designed into the specific discipline, in which
case the DF of the discipline will be lower than normal. If it is used
as a separate aid the focus gives \[ 2\%/rank\] in the focus. 

\subsection{Mental movement}

Mental movement governs that distance in M/sec that can be covered by 
a psionics powers while concentrating. This is subject to the specific
rules of a given psionic skill. So typically a Telekinetic can move
objects at a rate equal to their mental movement.


\section{description}

Psionic Disciplines have a series of attributes

\begin{description}

    \item[Penetration]

    Describes how well the discipline overcomes resistance. Usually only
    important for attack or healing disciplines.

    \item[Quantitative Effect/QE]

    What the discpiline does in terms of mass moved, healing effect,
    etc.

    \item[Difficulty Factor/DF]

    The difficulty associated with the given discipline.

    \item[Energy Cost/EC]

    The amount of mental fatigue used to perform the discipline.

    \item[Cast Time/CT]

    The time needed to perform the discipline.

    \item[Range/RNG]

    Range of efect (could be 0)

    \item[AOE]

    Area of effect (could be 0)

    \item[Visibility]

    How easily detected the discipline is.

    \item[Integrity]

    How well the discipline will endure without direct concentration.
    Usually translates to duration.

\end{description}

\section{Skills Involved}

There is a Filter Skill for each of the areas of psionics. 
In addition there is a Perception skill associated with each area.

And finally there a specific skills within each area of psionics.

A Telekinetic must have the following range of skills:

\begin{description}
    \item[Telekinetics(Filter)]
    \item[Telekinetic Perception]
    \item[Telekinesis]
\end{description}

A Telekinetic could have the following other skills (as an example)

\begin{description}
    \item[Discipline:Levitate]
\end{description}

A Biokinetic must have the following range of skills:

\begin{description}
    \item[Biokinetics(Filter)]
    \item[Biokinetic Perception]
    \item[Biokinesis]
\end{description}

A Biokinetic may have the following range of skills (as an example):

\begin{description}
    \item[Gestalt Lore:My Own Body]
    Lore absorbed via experience with direct mind manipulation of the
    given body.
    \item[Gestalt Lore:Human]
    Lore absorbed via experience with direct mind manipulation of the
    given type of body.
\end{description}

A Temporokinetic must have the following range of skills:

\begin{description}
    \item[Temporokinetics(Filter)]
    \item[Temporokinetic Perception]
    \item[Temporokinesis]
\end{description}

A Temporokinetic may have the following range of skills (as an example):

\begin{description}
    \item[Discipline:Teleportation]
    \item[Memorize Location]
\end{description}

A Energetic must have the following range of skills:

\begin{description}
    \item[Energetics(Filter)]
    \item[Energetic Perception]
    \item[Energetics (Transform Energy )]
\end{description}

A Energetic may have the following range of skills:

\begin{description}
    \item[Knowledge of Energy:Heat or]
    \item[Physics E\&M or]
    \item[Gestalt Lore:IR]
    Lore absorbed via experience with direct mind manipulation of the
    given energy frm.
\end{description}

A Telepath must have the following range of skills:

\begin{description}
    \item[Telepathics(Filter)]
    \item[Telepathic Perception]
    \item[Telepathy]
\end{description}

A Telepath may have the following range of skills:

\begin{description}
    \item[Discipline:Empathy]
    The ability to sense moods and feelings
    \item[Discipline:Telempathic Projection]
    The ability to project moods and feelings onto another entity
    \item[Discipline:Surface Read]
    The ability to read surface thoughts.
    \item[Discipline:Telepathic Surface Projection]
    The ability to insert surface thoughts into another's mind.
    \item[Discipline:Telepathic Read]
    The ability to read thoughts.
    \item[Discipline:Telepathic Projection]
    The ability to insert thoughts into another's mind..
    \item[Discipline:Telepathy]
    The ability to communicate in coherent thought patterns with another
    entity.
    \item[Discipline:Mind Meld]
    The ability to coprocess/share minds with another entity. This
    includes sharing all of another's senses.
\end{description}

A Mystic must have the following Skills

\begin{description}
    \item[Mystokinetics(Filter)]
    \item[Mystokinetic Perception]
    \item[Mystokinetics]
\end{description}

A Mystic may have the following Skills

\begin{description}
    \item[Discipline:Precognition]
    The ability to see into the future
    \item[Discipline:Retrocognition]
    The ability to see into the past.
    \item[Discipline:Object Reading]
    The ability to see into an object's past.
\end{description}

Any psionic may typically have these support skills:

\begin{description}
    \item[Stealth(Psionic)]
    The skill associated with trying to hide that psionics are being
    used. Ofetn used to hide from another psionic when being
    searched for. 
    \item[Deception(Psionic)]
    The skill associated with trying to disguise the psionic action being
    performed. Often used to disguise one type of action for another.
    \item[Focus:Crystal]
    Skill in using something as a focus for a discipline. See above.
    \item[Meditation]
    Skill in tuning the rest of the world out in order to better
    concentrate on what is being done.
\end{description}

\section{How it should feel}

These things should be possible for a Telepath in ideal conditions (at
rest , meditating, setup, with some sort of focus):

\subsection{Telepathics}

Telepaths at a low level should be able to pick someones mind
pattern out of a crowd in a room, 'path to someone in
the room if they don't know them, 'path to someone anywhere 
in the ship if they know them well, and maintain that conversation
for a few (10-20) minutes. Assuming, of course, a non-hostile partner.

Telepaths at a medium level (10-12) of skill should be 
able to locate a mind somewhere in a ship, 'path to someone in a 
ship if they don't know them, 'path to someone anywhere in a town 
if they know them well, and maintain that conversation for 
a few minutes.

Telepaths at a high level (20) of skill should be able 
to locate a mind somewhere in a city (takes time), 'path to someone
in a town if they don't know them, path to someone in orbit if they know
know them well.

Range is less of an issue than familiarity and willingness of partner.
Empathy is the detection of mood and feelings. 

\subsection{Temporokinetics}

Temporokinetics at a low level should be able to sense when a teleport 
has occured in the area, 'port small objects in line of sight or to
nearby (a few meters) recently memorised locations. At a medium level (10-12) 
they should be able to teleport themselves to line of sight locations. At 
high levels (14-18) they should be able to teleport themselves to memorised 
places beyond line of sight. At rank 20 it should be possible to teleport to 
and from orbit.

Knowledge of the area targeted is critical to success. It is far more
difficult to teleport another person than it is for the psionic to teleport
themself.

\subsection{Biokinetics}

Biokinetics at a low level should be able to ease pain, sense what is wrong 
with themselves or others. At a medium level they should be able to cure shock,
bring down fever, and anaesthesize people. At a high level you should be able 
to speed reflexes, increase strength, Suspend animation. 

Knowledge of the body in question is a key factor. Biokinetics often
have ranks in knowledge of their body and ranks in inner awareness. 

\subsection{Telekinetics}

Telekinetics at a low level should be able to move chairs across a room, 
float cups of coffee. Medium level telekinetics should be able to levitate 
themselves. High level 'kinetics should be able to do simple flight
with themselves, block. Master level 'kinetics should be able to 
do fairly quick flight, and maybe even block punches.

Most characters will need to see the objects being manipulated in order
to manipulate them.

\subsection{Energetics}

Energetics at a low level should be able to heat/cool cups of water,
keep themselves warm. At a medium level they should be able to ignite
paper or wood, cause electrical switches to bridge, blow light bulbs.
High level energetisists should be able to redirect
spark gaps, protect themselves from UV. Master levels should be able to 
generate light, protect themselves from radiation.

Knowledge of the type of energy being manipulated is the key factor.
That knowledge may occur as a abstract model (scientific knowledge), a
symbolic paradigm (Wiccan elemental worldview) or a pragmatic experience.

\subsection{Mystokinetics}

Mystics at a low level should be able to get a general sense of danger.
Medium level should be able to determine that this person is in 
danger (if they have met them). High level mystics should be able to 
predict simple events. Master level mystics should be able to tell 
you what horse to bet on.


Mystokinetic skills are tricky. There are several degrees of freedom
in any attempt to precog: The accuracy of detail - what happened, The
accuracy of Presentation - How it looked, and Depth - Background or
context. This can also be What, When, WHere, Why, How, and Viewpoint.

Different psionic usage cultures have different relationships to the
mystokinetic degrees of freedom. 

\end{document}
