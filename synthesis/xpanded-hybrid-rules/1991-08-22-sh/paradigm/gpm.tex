\chapter{General Playing Mechanics}

\section{Introduction}

This chapter discusses various critical pieces of the game system that 
effect just about every character. They are not specific to either 
combat or non-combat situations. 

In situations that involve any type of conflict, whether physical, 
mental or verbal, the decision on who acts first may be 
critical\footnote{ Note in the design notes that we have intentionally 
removed first action determination from the Combat section}.

The total model of Space Hybrid centers around an reaction/action sequence.
The character determines when his reaction will occur and at that time 
declares his action. The action speed is added to the time of the reaction 
and the total is when the actual action finishes. Once the action has 
occurred the character rerolls initiative unless he is acting on a
preset action. 

\section{Time Scale}
\INDEX{Time Scale}{Time scale}

Time is typically broken down into the following common units:

% Include Time Scale Table
% FILE Time Scale Table
% REF 
\begin{table}[h]
\caption{Time Scale}
\centering
	\begin{tabular}{ll} \hline
	Pulse      & 1/5 Second	\\
	FivePulse  & 1 Second   	\\
	TenPulse   & 10 Pulses = 2 seconds	\\
	Round      & 50 Pulses = 10 seconds	\\  \hline
	\end{tabular}
\end{table}


In general the more tense or critical an action is, the smaller the unit
of time that is used by the GM. 

\section{Perception}

At the point where an action occurs that may require action on the part
of the player a perception roll is made.

\section{Initiative or Reaction}
\INDEX{Initiative}{Initiative}

Initiative is broken down into two areas: engaged and unengaged 
initiative. Unengaged initiative is rolled when the individuals 
involved are first aware (i.e. Focusing on each other) of each other but or
when this is the first initiative roll to be made by a character. 

Engaged initiative is rolled at any other time.

\subsection{Unengaged Initiative}
\INDEX{Unengaged Initiative}{Initiative}

Unengaged Initiative rolls are merely GAW, PAW, or MAW rolls made 
against the appropriate EF. 

There are, of course, modifiers to this roll as detailed in table 
~\ref{tab:UnengagedInitiativeMods}

The speed of reaction is given by dividing the SN of the roll by 10.
This speed is subtracted from 10 and that is how many counts it takes 
the character to react.

% Initiative Roll Modifiers
% FILE Initiative Roll Modifiers
% REF tab:UnengagedInitiativeMods
\begin{stable}{Perception Modifiers}{ll}
\label{tab:PM}
	Situation				&  DF \\ 
\TableSubtitleRule
	Blinded				 &  -5 \\
	Deafened				 &  -3 \\
	Drunk/Stoned			 &  -5 \\
	Asleep					 &  -4 \\
	Poor Lighting			 &  -3 \\
	Not Alert 				&  -3 \\
	Alert					&  0  \\
	Actively Watching		& +3  \\ 
\end{stable}


The concept of suprise as such does not really exist as a seperate 
state. A character who is suprised is one who was not actively 
watching who got a poor roll at EF -3. There is a situation associated 
with poor Unengaged Initiative rolls called Default Actions. See 
section \ref{sec:DefaultActions}

\subsection{Engaged Initiative}
\INDEX{Engaged Initiative}{Engaged Initiative}

The most basic engaged initiative is gotten by rolling 2d10 and 
subtracting the character's speed and his speed for the rank he has 
in the skill he is using.

\subsubsection{Character Speed}
\INDEX{Character Speed}{Character Speed}

The characters speed is taken from table~\ref{tab:CharSpeed} on page~\pageref{tab:CharSpeed}
. In reality, the character has a range of speeds depending on the 
given modifiers, but most situations will require the speed for EF =
7.

% FILE Initiative Roll Table
% REF tab:CharSpeed
\begin{table}[h]
\centering
\caption{Character Speed Table\label{tab:CharSpeed}}
\small
	\begin{tabular}{llllllllllllll} \hline
	SB & \multicolumn{13}{c}{Ease Factors} \\ \hline \hline
	   & -2  & -1  &  0  &  1  &  2  &  3  &  4  &  5  &  6  &  7  &  8  &  9  & 10 \\ \cline{2-14}
	 3  & -8  & -6  & -3  & -3  & -2  & -2  & -2  & -2  & -1  & -1  & -1  &  0  &  0 \\
	 4  & -8  & -6  & -3  & -3  & -2  & -2  & -1  & -1  & -1  &  0  &  0  &  1  &  1 \\
	 5  & -8  & -6  & -3  & -3  & -2  & -2  & -1  & -1  &  0  &  1  &  1  &  2  &  2 \\
	 6  & -8  & -5  & -3  & -2  & -2  & -1  & -1  &  0  &  1  &  1  &  2  &  2  &  3 \\
	 7  & -8  & -5  & -3  & -2  & -2  & -1  &  0  &  1  &  1  &  2  &  3  &  3  &  4 \\
	 8  & -7  & -5  & -3  & -2  & -1  & -1  &  0  &  1  &  2  &  3  &  3  &  4  &  5 \\
	 9  & -7  & -5  & -3  & -2  & -1  &  0  &  1  &  2  &  2  &  3  &  4  &  5  &  6 \\
	10  & -7  & -5  & -3  & -2  & -1  &  0  &  1  &  2  &  3  &  4  &  5  &  6  &  7 \\
	11  & -7  & -5  & -3  & -2  & -1  &  0  &  1  &  3  &  4  &  5  &  6  &  7  &  8 \\
	12  & -7  & -5  & -3  & -2  & -1  &  1  &  2  &  3  &  4  &  5  &  7  &  8  &  9 \\
	13  & -6  & -5  & -3  & -2  &  0  &  1  &  2  &  4  &  5  &  6  &  7  &  9  & 10 \\
	14  & -6  & -5  & -3  & -2  &  0  &  1  &  3  &  4  &  5  &  7  &  8  & 10  & 11 \\
	15  & -6  & -5  & -3  & -2  &  0  &  2  &  3  &  5  &  6  &  8  &  9  & 11  & 12 \\ \hline
	\end{tabular}                                                                                                                     
\normalsize
\end{table}                                                                                                                         


\subsubsection{Speed Gains Due to Rank in a Skill}

The character may add \( Rank/3 \) points to their speed when using a
skill. This may only be done once the character has decided to use a
given skill. 

\section{Preset Reactions}
\INDEX{Preset Reactions}{Preset Reactions}

A character may decide to preset a reaction. A preset reaction is 
attempt to make yourself sensitive to a specific stimuli to the exclusion
of all other stimuli. The advantage is that it allows an additional EF +2 
to an unengaged initiative roll. The disadvantage is that it adds EF 
-3 to any other perception roll. Gunfighters waiting on someone elses 
draw of a weapon would preset a reaction. 

A character does not have to be actively watching in order to preset 
a reaction.

A Preset reaction may only be held for MST in the time scale 
that the players are working in before a cost of 1 MFT must be 
expended.

An engaged individual gets to roll 1d10 rather than 2d10 for 
initiative.

\begin{quote}
Isaac (GAW = 20) expects to be attacked while walking in the alley. 
He states that he is actively watching for attack and , when he is 
attacked, he will dodge. When he is attacked he gets to roll 
initiative.  

Since he is actively watching he gets an EF +2 modifier to his 
Unengaged Initiative roll in addition to the EF +2 modifier for the preset. His 
total is \( EF = 7+2+2 = 11\). His awareness is currently a 110\% and
he rolls a 51. His SN is 59, or a 6 speed. So he reacts in \( 10 - 6
= 4 \) counts or 4/5 of a second. So at count 4 he can start a dodge.
\end{quote}

\begin{quote}
Reginard (PAW = 18) is tucked around a corner in the alley listening 
for someone to mug. He is using only his hearing and is actively 
listening. When he hears someone just around the corner he is going 
to jump around the corner and slash at him with a dagger.

Because he is actively listening he gets EF +2. Because he is 
concentrating on PAW he gains an additional EF +2. Because he is 
using a preset he gets an EF +2. The total is \( EF = 7+2+2+2 = 13\).
His awareness for EF 13 with a SB of 9 is 117\%. 

He rolls a 64. His SN is 53 or 5 speed points. He reacts in \( 10 - 5
= 5\) counts so he can start his attack 5 counts after he hears a
noise.
\end{quote}

\begin{quote}
Reginard, upon leaping out from the corner starts his attack. The 
dagger he is using has a speed of 2 so his attack occurs at count \( 
5 + 2 = 7 \). 

Isaac, upon hearing the noise starts his dodge at count 4, it has a
speed of 3 counts so he completes his dodge at count 7. 

The attack fails ( we are not showing the attack resolution process 
here) and both must reroll initiative. Both roll 2d10 and subtract 
their speed for EF 7. If Reginard continues to use his dagger he will 
be able to subtract his speed bonus for his rank in dagger. 

\end{quote}
\section{Actions}
\INDEX{Actions}{Actions}

Actions normally begin at the count given by the initiative roll. The 
decision about what action is to be performed, if not already made,  
must be made at this point. The speed of the action is determined and 
the character takes this action on a pulse given by Initiative +
Action Speed.  

\begin{quote}
Rashid "the Twitch" (GAW=24) is walking past an alley when he notices the 
altercation in the alley. He is not actively watching (EF -3), so he 
rolls on the EF= 4 column. So his awareness is \( 4 \times 12 = 48\% 
\).  Rashid rolls an 80 and thus his initial speed is -32/10 or -3. 
He starts his reaction \( 10 - -3 = 13 \) counts after hearing the 
noise. 
\end{quote}

\section{Speeds of Actions}

Most actions have a speed associated with them. All simple actions 
, unless otherwise noted, have a standard speed of 5 counts. 
 
% Speeds of Basic Actions
% FILE Speeds of Basic Actions
% REF 
\begin{table}[h]
\centering
\caption{Speeds of Basic Actions}
	\begin{tabular}{ll} \hline
    Action						& Speed \\ \hline
	Lift Light object			&	5 \\
	Lift Heavy Object           &  20 \\
	Any Simple Physical Action  &   5 \\
	Perception					&	5 \\ \hline
	\end{tabular}
\end{table}


\section{Default Actions}\label{sec:DefaultActions}

\section{Speeding up Actions}

Speeding up an action to 1/2 as long, causes the action to have 1/2 the 
EF. Round to the Worst. At no point can an action take less than 1/3 
of its base speed.

\section{Drawing a Tool or Weapon}

This most often applies to drawing a weapon but can also apply to 
other tools.

In general, when a weapon is in hand, all normal weapon speeds apply. 

In order to get a weapon into ones hand it takes \( 2 \times 
Speed_{weapon} \) in pulses. 

In order to get a weapon in hand faster than \( 2 \times 
Speed_{weapon} \) requires a fast draw or ready roll against the weapon's 
skill. A successful ready roll brings the tool or weapon to bear at 
\( Speed_{weapon} \).  

\begin{quote}
When count 18 comes around Rashid attempts to fast draw his Isaac 
starts his dodge on 9, the speed of the dodge is 5 counts so 
Isaac starts his dodge on 9 and ends it on count 14. 
\end{quote}
 

\section{Resolving an Action}
Actions usually require only a skill roll to be made. 

\begin{quote}
Isaac starts his dodge on 9, the speed of the dodge is 5 counts so 
Isaac starts his dodge on 9 and ends it on count 14. 
\end{quote}

\section{Multiple Actions}

For each ``Task'' there are a number of actions that can be performed. 
If the actions involved in the task are simple the GM may choose to
resolve  them with one roll rather than resolving each individual
action. This may best  be done when each action is simple ( i.e. EF >=
6). 

Jogging across the street and leaping a small fence is something suited
to being a task. Normally though, the GM would not normally even ask
the character to roll a task roll simply because the actions are all
very easy. But if the character stands the chance of being exposed to
someone looking for him then a roll should be made. 

The EF for the ``Task'' Roll is given by \[ EF_{Task} = 5 - ( 1/2
\times {Number\ of\ Actions }) \]  The Gain Value for task roll is
excellent success. This is a``Shifted Result'' roll. It is shifted to
the less severe result. A Normal failure is a success, a Normal
Critical Failure is a Normal Failure, etc \dots

The experience points gained from a task roll match whatever the
final roll is. Those points may be distributed among any of the skills used
in the task.

\section{PFT Cost of Actions}

An individual can use a great deal of energy in performing actions in
combat or doing other simple tasks. For each period of activity {\em
when the activity is over} the cost of the character's activity is
subtracted from the PFT of the character.

\footnote{What are the time interval breakdowns for PFT loss} 

% PFT Loss Numbers for a given activity
% FILE PFT and PEX cost for a given activity
% REF 
\begin{table}{PFT and PEX costs for activity}
	\begin{tabular}{lcc}

	Activity		&	PFT & PEX \\	
\hline
	Crawling		&	   &	  \\
	Walking			& 1/min	& 6/hr \\
	Jogging			&		& 1/min \\
	Running			&	   & 6/min \\
	Dash			&		& 2/sec \\
	Chopping Wood	 & 3/min & 18/hr \\
    \end{tabular}
    \caption{PFT and PEX costs for activity}
\end{table}


\section{MFT Cost of Actions}

An individual can use a great deal of energy in performing actions in studying or
doing other simple tasks. For each period of activity {\em when the 
activity is over} The cost of the activity is subtracted from the 
MFT of the character.

% MFT Loss Numbers for a given activity
% FILE MFT and MEX cost for a given activity
% REF 
\begin{SHTable}
	\begin{tabular}{lcc}

	Activity		&	MFT & MEX \\	
\hline
	Studying		  & 1/min & 6/hr \\
	Spell Research	& 3/min & 18/hr \\
    \end{tabular}
    \caption{MFT and MEX costs for activity}
\end{SHTable}


\section{Physical Movement}
\INDEX{Physical Movement}{Physical Movement}

Each character has a statistic named Physical Movement. This is the character's
movement in meters/second at a dash. There are a total of five different types
of movement that a character may utilize. Each type of movement has its own
movement rate which is derived from the character's movement statistic. 
Ideally the player will have the full range of movements listed on his 
character's sheet. 

% Include Movement Types Table
% FILE Movement Types
% REF 
\begin{table}[h]
\centering
\caption{Movement Types}
	\begin{tabular}{ll} \hline
	Movement Type		& Rate of Movement (meter/second) \\ \hline
	No Move             & \( 0 * Movement \) \\
	crawls, slow walks 	& \( 0.50 * Walk  \) \\
	Walking             & \( 0.50 * Jog   \) \\
	Jog					& \( 0.50 * Run   \) \\
	Run                 & \( 0.50 * Dash  \) \\
	Dash                & \( 1.00 * Movement \) \\ \hline
	\end{tabular}
\end{table}


\begin{quotation}
As an example, Joe Daring has a PST of 16 and a PAG of 16. His movement is
\( {{(16 + 16)} \over 4} = 8 \) . So Joe can Dash at 8 m/sec, run at 4 m/sec, jog
at 2 m/sec, walk at 1m/sec and crawl at .5 m/sec.
\end{quotation}

If the movement is being resolved during a time scale of greater than every
pulse one can get the distance traveled by simply multiplying the movement 
of the individual times the time spent moving. The time spent accelerating
is ignored as being negligible.

\begin{quotation}
Let us say that Joe Daring spends 15 seconds running down a deserted street.
If he doesn't run out of street he will have covered 4 * 15 = 60 meters. If 
this seems a bit short, keep in mind that a run is not a full dash. At a full
dash Joe would have covered twice the distance and would be slowing down pretty 
drastically due to losing wind.
\end{quotation}

\subsection{Acceleration}
\INDEX{Acceleration}{Acceleration}

\marginpar{It is important to remember that the accellaration rules 
should only be used when the distance travelled by the characters 
over a {\bf short period} of time is important }

In dealing with movement on a pulse by pulse scale we need to actually
deal with acceleration. The sequence is quite simple. Whatever the
final movement  rate is that the character intends to use is considered
the target movement rate. When the character first starts moving he
makes an acceleration roll in  order to start moving at the movement
rate just below the target movement rate. Once the roll is made the
character is now moving at that lower rate. On his next  initiative the
character may attempt to accelerate to the target movement. Note  that
the gain number is the movement rate. If an acceleration roll is failed
the end result is that the character drops to the next lowest available
movement  rate. Of course each of these acceleration rolls has its own
EF modifier.  

% Acceleration Modifiers
\input{tgpm3}

\begin{quotation}
Reed Johnson has a movement of Dash 10, Run 5, Jog 2.5, Walk 1.3, Crawl .6
\end{quotation}

\subsection{Movement Modifiers}

% Targeted Action Movement Modifiers
% Targeted Action Movement Modifiers
\begin{table}[htb]
\centering
\caption{Targeted Action Movement Modifiers}
	\begin{tabular}{||l|l||} \hline
	Slow move: crawls, slow walks (combat )   & EF -1 \\
	Normal move: Walking                      & EF -2 \\
	Double Move: jog                          & EF -3 \\
	Triple Move: Run                          & EF -4 \\
	Fast move: Dash                           & EF -5 \\
	Vehicular Movement                        & EF -6 \\  \hline
	\end{tabular}
\end{table}


\section{Mental Movement}
\INDEX{Mental Movement}{Mental Movement}
This is a measure of the character's speed of mental travel. It is 
usually only used in Psionics and Computer usage.

\section{Opposing Skill Rolls}

An opposing skill roll in a roll in which the character attempts to 
undo an action done previously by another character. Typically the SN 
of the original action is taken as a negative modifier to the current 
skill roll.

\section{Stealth and Concealment}
\INDEX{Stealth and Concealment}{Stealth and Concealment}

Opposing Skill Rolls

\section{Deception and Detection}
\INDEX{Deception and Detection}{Deception and Detection}

Opposing Skill rolls

