\chapter{Psionics}

\section{Outline}

This chapter outlines one of the more interesting areas of SF gaming: 
Pyschic Phenomenom or Psionics. What is to be noted is that no 
comfortable way could be found to make a truly generic set of psionic 
abilities. A great deal of the limitations described here are indeed 
implementation specific and could be changed on GM's whim.

\section{Definitions and Description}

Psionics is the application of mental force to directly affect an entitie's
surroundings. The use of psionics is dependent upon the total concentration
of the psionic. 


\section{Areas of Psionics}

Energetics, Telekinetics, Temporokinetics, Phanokinetics, Biokinetics

\section{Gaining Psionics}

All entities have the potential for psionics. 
All Psionic Skills are mental disciplines.
To determine the character's area of talent roll on table

\section{Usage of Psionics}

There are several things of note to be mentioned. The first and foremost
is that it is always easier to affect inanimate objects with Psionics.

When attempting to affect other entities the percentage chance to succeed
is lowered by the defending entities MDF. An actively defending individual
lowers it by 2 * MDF.

The second thing of note is that even affecting yourself is somewhat
difficult. A Roll must be made to lower MDF.

\subsection{Skill Rolls}

Attempting to use Psionics requires a skill roll just as any other skill.
The same EF modifiers apply as to any other action. 

\subsection{Concentration Checks}

If the Psionic is disturbed in any way the entity must make an EF 2 roll
against that psionic skill in order to continue their concentration. This
concentration may be made up to three times per pulse. If more than that is
needed , give it up. Psionics may use special aids in concentrating including: 
Meditation, Ear Plugs, Drugs etc...

\subsection{Mental movement}

Mental movement governs that distance in M/sec that can be covered by 
a psionics powers while concentrating.

\subsection{Range of Psionics}

The range on Psionics is typically equal to the MST in meters.

The character may choose fixed ranged when he declares the psionic 
action or he may declare for cost based range. 

\subsection{Damage of Psionics}

The base "damage" comes from the Mental Strength Effect (MSE).

\subsection{Base effect}

Cost based \\
or Fixed\\

\subsection{Knowledge}

No Knowledge(Instinctual usage) \\
Trained Only(No Training) \\
Everyone is trained \\


\subsection{Drain Rolls}

After a psionic has used a psionic skill he rolls a 
"drain" to see if he expends the cost of the psionic skill. A drain
roll is an EF 2 roll against the skill. Someone who fails the psionic 
skill roll will take an EF -1 to the drain roll, a critial failure of 
the skill roll causes an EF -2 on the drain roll. Of course, you can
critically fail a drain roll. 

Psionics are described and treated in a manner similiar to Thrown weapons.

\begin{figure}[hb]
\centering
\caption{Psionic Format}
	\begin{description}
	\item[Name] Self Explanatory
	\item[Cost] The cost in MEN for using the skill.
	\item[Psionic Accuracy Decrement] 
	\item[Psionic Damage Multiplier]
	\item[Psionic Damage Decrement]
	\item[Experience Point Multiple]
	\end{description}
\end{figure}

\section{Psionic For\-mat Ex\-pla\-nation}

\begin{description}
	\item[Name]
	Self Explanatory
	\item[Cost]
	Usually the cost in MFT.
	\item[Psionic Accuracy Decrement (PAD)]
	The loss in accuracy for a given distance
	\item[Psionic Damage Multiplier (PDM)]
	The multiple times the users base that is used to produce a characters
	damage.
	\item[Psionic Damage Decrement (PDD)]
	The loss in dam\-age as\-so\-ciated with range in a psionic attack
	\item[Experience Point Multiple (EPM)]
	The Experience point cost necessary to raise the skill
\end{description}

\subsubsection{Areas of Psionics}

\begin{description}
	\item[Telekinetics]
	Telekinetics is the use of the mind to impart kinetic energy to an
	object. The use of psionics to toss objects around is a telekinetic 
	action.
	\item[Temporokinetics]
	Temporokinetics is the use of the mind to affect the spatial integrity
	of the surrounding area. Teleportation is a temporokinetic action.
	\item[Biokinetics]
	The use of a mind to directly influence a biological system. Placing 
	oneself in suspended animation is a typical biokinetic feat.
	\item[Energetics]
	The use of the mind to directly affect the molecular nature of a
	material. Setting fire to paper or dissolving a glue would be 
	typical actions of an energetic.
	\item[Phanokinetics]
	The use of Psionic power to increase an enitities sensitivity to the 
	patterns and gestalt of its environment. Telling when a entity lies 
	due to relatively subtle logic flaws in a statement is a typical 
	Phanokinetic action.
\end{description}


