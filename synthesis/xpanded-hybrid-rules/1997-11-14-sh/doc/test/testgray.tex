%this file is an example of using postscript specials to give 
% headings a gray background  

\hsize=5.7in 
\font\rm=ptmr at 10pt
\font\norm=ptmr at 10pt
\font\tenrm=ptmr at 10pt
\font\twelverm=ptmr at 12pt 
\font\twelvebb=ptmb at 12pt
\font\fourteenbf=ptmb at 14pt

\nopagenumbers

\newdimen\grayht \newcount\grayheight 
\def\getgrayheight#1#2#3{\grayht=#1bp \grayheight=#1  \setbox0=\vbox{#3} 
\loop \ifdim\grayht < \ht0
    \advance \grayht by\baselineskip \advance \grayheight by #2 \repeat}

\def\graybar#1#2#3#4#5{\special{"gsave   #1 setgray #2 #3 rmoveto #4 0 rlineto  0 -#5 rlineto 
-#4 0 rlineto 0 #5 rlineto  closepath fill grestore"}}

\def\grayhead{\graybar{.8}{-6}{13}{216}{\the\grayheight}}
\def\graybhead{\graybar{.7}{-28.8}{15}{12}{\the\grayheight}}
\def\graythead{\graybar{.8}{-6}{15}{380}{\the\grayheight}}
\def\grayahead{\graybar{.7}{-28.8}{13}{12}{\the\grayheight}}

\def\heada#1{{\hsize=210bp\vskip 24pt plus6pt minus4pt
{\twelvebb\parindent=6pt \baselineskip=14bp
\getgrayheight{20}{14}{#1}\item{}\grayhead #1\par}
\vskip 24pt plus 6pt minus 4pt}}

\def\startart#1#2{{\fourteenbf\baselineskip17bp\raggedright\hyphenpenalty=9000
\parindent=.4in\getgrayheight{22}{17}{#1}\item{}\graybhead\graythead#1\par}
{\twelverm\vglue 12pt plus6pt\parindent=.4in\baselineskip=14bp 
\getgrayheight{20}{14}{#2} \item{}\grayahead #2\par} \vskip 4pt }

\startart{A Test of Inserting Postscript Code into \TeX\ for
Graybars -- use with dvips}{Thomas Dunbar, ClearVu Technical Writing
{\tt\quad Bitnet:gstd at vtvm2}}           \hsize=4in

\heada{1. Caveats\hfil\break and Cautions}

Warning:  your print shop may not be able to use your pretty slick since
they may have difficulty photographing the graybar! The solution is to have
the linotronic print directly to film.

This file makes use of Arbortext's special for dvips: the current dvi
location is made the postscript {\tt currentpoint} and the default
postscript coordinate system and scaling (unit=bp) are used.

\heada{2. Code}

The coding is straightforward postscript except that the height of
the box is calculated via \TeX.  If needed, the width could be
computed similarly. Note that the bar is to be painted first so the
text can overlay it.

\bye
